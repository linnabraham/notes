\documentclass{../template/texnote}

\title{Quantization Of Energy Part 1: Black-body Radiation and the Ultraviolet Catastrophe}

\begin{document}
    \maketitle \currentdoc{note}
    %<*note>
\url{https://www.youtube.com/watch?v=7BXvc9W97iU}
\\

3.25

Heat is the transfer of kinetic energy from one place to another.
In a solid, there is no translation motion.
So in a solid hot metal it takes the form of atomic vibrations or oscillations.
These vibrations are what generate the light we see in the black body spectrum.\unsure{So if it were a gas, there could be translational, virational and rotational motions possible.}
Planck proposed that the vibrational energies of these atoms are quantized.
And by extension the energies of the electromagnetic waves emitted by the atoms must be quantized.
    %</note>
    \printbibliography
\end{document}
