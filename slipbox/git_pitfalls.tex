\documentclass{../template/texnote}

\title{Git Pitfalls}

\begin{document}
    \maketitle \currentdoc{note}
    %<*note>

	\begin{itemize}
		\item The \texttt{git branch --merged} command checks the merged status of existing branches with respect to the current checked out branch. It doesn't show the list of all merged branches.
		\item In git, there are two types of dates: \texttt{author-date} and \texttt{commit-date}. \texttt{git log} shows the \texttt{author-date} by default wheras GitHub shows the \texttt{commit-date} by default. Since the author-date is part of the patch format, you can get it from appending .patch while viewing a commit on the GitHub website.
		\item The \texttt{git stash pop} command also applies the stash onto the current checked out branch. If the stash is made from one branch but popped on another, unexpected things might happen.
		\item \texttt{git cherry-pick} might fail if you have uncommited changes even if the commit you are trying to pick doesn't involve that specific file. Do a clean stash to stash all working directory changes before attempting.
	\end{itemize}
    %</note>
    \printbibliography
\end{document}
