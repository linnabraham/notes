\documentclass{../template/texnote}

\title{Suns Spectrum}

\begin{document}
    \maketitle \currentdoc{note}
    %<*note>
In what ways does the observed spectrum of the Sun (both on Earth and from Space) deviate from an ideal black body spectrum and why?

%When talking about the black body spectrum, the object in question has a single temperature?
\textcolor{teal}{What does the deviations in the observed solar spectrum teach us?}

Refer \citetitlebyauthor{shu_physical_1982}

``The agreement is reasonably good from the ultraviolet to the infrared. Discrepencies arise in the X-ray, far UV and radio portions of the electromagnetic spectrum, and these discrepencies are especially noticeable during flare and burst activity in the Sun.
Even gamma rays are detectable during large bursts on the Sun.
However the total energy contained at these wavelengths is small, especially in the Quiet Sun.
Near the peak of the Sun's emission, we can reasonably claim that a 5,800K blackbody gives a fair representation of most of the output of the Sun.\unsure{Why is this true and what does this imply?}
''

Even at  optical wavelengths there is considerable departure in the SED of the continuum radiation.
``These clues suggest that the Sun is not at a uniform \unsure{Does he mean uniformity in each layer or is uniformity required throughout} temperature and its surface(or atmospheric)\unsure{Why only mention surface layers and why not interior layers?} layers are not in perfect thermodynamic equilibrium with its radiaiton field. (Otherwise, the emission would satisfy Planck's law.) The temperature of 5,800 K must evidently characterize only part of the surface layers.''

``If the Sun is not at a uniform temperature, is it hotter or cooler in its interior than its surface?''

Two observations, one, that of limb darkening and other, presence of absorption lines in the spectrum points to the former.\info{Explain how}

    %</note>
    \printbibliography
\end{document}
