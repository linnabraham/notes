\documentclass{../template/texnote}

\title{The Sun}

\begin{document}
    \maketitle \currentdoc{note}
    %<*note>
%\section{Approach}
%How can we approach the subject of solar physics?
%\begin{itemize}
    %\item How do you  we know the things that we know?
    %\item What do we need to know?
    %\item What are the known knowns and known unknowns?
%\end{itemize}
%\section{Books}
%Books that I have read for the purpose of making these notes.
%\begin{itemize}
    %\item \citetitlebyauthor{choudhuri_natures_2015}
    %\item \citetitlebyauthor{shu_physical_1982}
    %\item \citetitlebyauthor{priest_magnetohydrodynamics_nodate}
%\end{itemize}
%\section{Sun and Astronomy}
%From the discussions in the previous chapter, it can be seen that the study of the Sun played a very important role in astronomy and the growth of science or scientific thinking.
Let's begin by asking some fairly basic questions about the Sun.
%\section{Questions}
%\begin{itemize}
    %\item How far away is the Sun?
    %\item How big is the Sun?
    %%\item What is the Sun's blackbody temperature?
    %\item Why does the Sun look yellow? 
    %\item What is the Sun composed of?
    %\item What is the source of the Sun's power?
%\end{itemize}

%\section{Remarks}

%When starting from first principles, it is possible to go along two routes.
%One the historical route and one the present route.
%%\citep{choudhuri_natures_2015}

\section{How far away is the Sun?}\unsure{Question}
People must have asked this question from a long time back.
The Greek philosopher correctly assumed that the Sun must be a infinitely large distance away. Or at least so far that the light rays coming from the Sun must be parallel.
He used this assumption to measure the Earth's radius.
This is indeed a very important question to ask. Because a number of physical parameters of the Sun can be derived once we know the distance to the Sun.
The size (diameter) of the Sun can be immediately calculated (from its angular size) once we know the distance to the Sun.\unsure{How can this be done?}
This distance also helps us calcuate the true luminosity of the Sun using the incident power of the Sunlight as measured on the Earth's surface.
This is because the radiated power falls off with distance following an inverse square law.
By assuming the Sun to be a blackbody, we can use this luminosity and the surface area of the Sun to estimate a surface temperature for the Sun.\unsure{What is the relation between luminosity, surface area and temperature?}
For a detailed understanding about how we know the distance to the Sun, refer Chapter 1 of the \citetitlebyauthor{mullan_physics_2022} in addition to the details in the Appendix \ref{knowing-sun}.

%\section{What is the Sun composed of?}
\section{How massive is the Sun?}\unsure{Question}
Since the volume of the Sun is now known, knowing the mass of the Sun would allow us to estimate an average density for the Sun.
This would in turn allow us to know what sort of material the Sun is made up of.
Is the Sun a solid body or is it a ball of gas?
The states of matter are usually distinguished by their densities.
%The answer turns out to be neither.
%The material in the Sun is in the fourth state of matter that is called plasma.
Would the surface temperature of the Sun be the same as it's internal temperature?
Knowing the mass of the Sun, we would be able to estimate the pressure at the Sun's centre.\unsure{How can we know this?}
And this pressure and the volume of the Sun would allow us to calculate the central temperature of the Sun.\improve{I might need to change this, because you need the assumption of an ideal gas before this can be done.}
For a detailed understanding of how to calcuate the mass of the Sun refer to chapter 1 of \citetitlebyauthor{mullan_physics_2022} and Chapter 5 of \citetitlebyauthor{shu_physical_1982}.

\section{What is the Sun made of?}\unsure{Question}
From the calculation of the pressure at the core of the Sun, it can be seen that the matter there exists in the fourth state of the matter, i.e., plasma.
\section{What powers the Sun?}\unsure{Question}
When a mass falls on Earth, its gravitational potential energy is converted to kinetic energy and heat.
Is a similar process at work in the Sun as well?
Calculations based solely on this gravitational potential energy was found to be insufficient.\unsure{How are these calculations made?}
Such processes predicted a lifetime to the Sun based on the current energy drain that was very less compared with the age of the Earth obtained from fossil evidence.
The answer to this riddle had to wait a long time until scientists came to know about the process of nuclear fusion.
The works of Albert Einstein, George Gamow, Hans Beth amongst others have contributed to our knowledge in this area.\unsure{How did they come up with the theory of nuclear fusion?}
For a better understanding of the source of the Sun's power refer to section 3.1 of \citetitlebyauthor{choudhuri_natures_2015} and Chapter 5 of \citetitlebyauthor{shu_physical_1982}.

\section{What Next?}
So far we have understood that the Sun is a huge ball of plasma that has a central temperature of 15 Million degrees Kelvin and located at a distance of 150 Million kilometers from the Earth.
Also that it is a star and like most other stars it is in that stage of its life where it produces heat and light by converting the hydrogen in it's core to helium though nuclear fusion.
One observation that has not yet been properly answered is -  what are Sunspots?

    %</note>
    \printbibliography
\end{document}
