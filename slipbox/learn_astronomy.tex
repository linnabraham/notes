\documentclass{../template/texnote}

\title{Learn Astronomy}

\begin{document}
    \maketitle \currentdoc{note}
    %<*note>
\section{Learn Astronomy}\label{learn-astronomy}

\begin{itemize}
\item
  What does resolution mean in astronomy?

  \begin{itemize}
  \item
    Refer:

    \begin{itemize}
    \tightlist
    \item
      \url{https://astronomy.swin.edu.au/cosmos/r/resolution}
    \item
      My article for airis mag
    \end{itemize}
  \end{itemize}
\item
  Blackbody radiation

  \begin{itemize}
  \tightlist
  \item
    What physical processes ensure that the blackbody spectrum consists
    of radiation of all wavelengths?
  \item
    Difference between emission lines and absorption lines in
    spectroscopy?
  \item
    Refer - ~``Absorption and Emission Lines''
    (\href{zotero://select/library/items/AP2G2R7N}{Laboratory, 1997,
    p.~35})
    (\href{zotero://open-pdf/library/items/M28PLAU2?page=35}{pdf})
  \end{itemize}
\item
  In astronomical context (observation) why is it said that the noise is
  square root of the number of photons?
\item
  What is the H-21 cm line in radio?
\item
  How can we distinguish between an active region that doesn't produce a
  flare and one that does produce?
\item
  How can we classify the areas of knowledge in astronomy?

  \begin{itemize}
  \item
    Based on wavelength regimes or objects of interest?

    \begin{itemize}
    \tightlist
    \item
      X-ray astronomy
    \item
      Gravitational wave astronomy
    \item
      Solar physics
    \item
      Galaxies
    \item
      Cosmology
    \item
      Radio astronomy
    \item
      Time domain astronomy
    \end{itemize}
  \item
    Based on Ranjeev's explanation, it is a cycle that consists of the
    following.

    \begin{itemize}
    \tightlist
    \item
      Observations ( Instrumentation)
    \item
      Identify the radiative process
    \item
      Compute physical parameters
    \item
      Do modelling (theorists)
    \item
      Make predictions
    \end{itemize}
  \end{itemize}
\item
  (Celestial) Coordinate systems

  \begin{itemize}
  \item
    Astronomical calendars
  \item
    Measuring distances (Cosmic ladder)

    \begin{itemize}
    \tightlist
    \item
      Variable stars as distance indicator or (Period-Luminosity)
      relation for Cepheids
    \end{itemize}
  \end{itemize}
\item
  Telescopes
\end{itemize}
    %</note>
    \printbibliography
\end{document}
