\documentclass{../template/texnote}

\title{What Is Astronomy}

\begin{document}
    \maketitle \currentdoc{note}
    %<*note>
%\section{Big picture of research in astronomy}
%(Following is inspired from a lecture by Ranjeev Misra.)
%\vspace{0.5cm}
%\newline
%Research in astronomy is an iterative process consisting of the following.
%\begin{itemize}
    %\item Observations
    %\item Identify the radiative process
    %\item (Find) physical parameters
    %\item  Modelling (Theorists)
    %\item Predictions
    %\item Instrumentation - observatory
%\end{itemize}

Astronomy is the study of celestial objects. To the primitive man, the celestial objects are the Sun, Moon and the stars.
What observations did he make about them?
What knowledge did he gain as a result of studying them?
The day and night cycle is one of the most obviously periodic event. Even the biological clock in our body adapts to this periodicity.
Then comes the periodicity of the seasons.\unsure{Why do seasons exists?}
Along with the seasons, the duration of day and night also changes.
The seasons and the day night duration changes are more pronounced as you move to the poles than if you are in the equatorial region.\unsure{Why is there this correlation}
People might have observed a star that doesn't change it's position throughout the year.
This star also known as the pole star always points to the north pole and could be used by sailors for navigating the deep seas.\unsure{Why do the other stars change their positions?}
The ancient people must have been apalled by the eclipses, not knowing when or how they appear. \unsure{How can you explain eclipses?}
With the growth of civilization more questions were asked.
Many of these questions had profound impacts on the world view held by these civilizations.
Even the divine status of the celestial spheres were at risk.
Among the stars people noticed some they called the planets or "wanderers".
They were to curious to know why these planets had such an odd motion.
The answers to such questions led finally to the Copernican revolution which fundamentally changed the world view. From the belief that the Sun and Moon orbited the Earth, we went to the Earth and other planets orbiting the Sun.
Some have gone so far as to ask the question - ``Why is the night sky dark?" (Olber's paradox). \unsure{How do we resolve Olber's paradox and what insight does it give?}
A more systematic study of the sky was made possible after the discovery of the telescope.
When Galileo trained a telescope towards the Sun, he noticed spots or blemishes on the Sun which would have been very unbecoming of a heavenly body.
Newton unified the laws of terrestial mechanics with the laws of celestial mechanics. That is he said that things are not any different in the heavens from how they are on the Earth.\unsure{How did Newton come to this conclusion?}
He also invented spectroscopy by splitting the Sunlight into it's compositional colours.
Fraunhoffer did this more systematically and found dark lines in the solar spectrum.
%\chapter*{Questions}
%\begin{itemize}
    %\item Why do the seasons form? Why does the duration of the day and night change during the year?
    %\item Why is the pole star useful as a navigational aid?
    %\item Why do eclipses occur?
    %\item What are the planets or wanderers? Why do they have this odd motion?
    %\item Why is the night sky dark? (Olber's paradox)
    %\item What are the blemishes on the surface of the Sun?
    %\item Why doesn't the moon fall towards the Earth like an apple falls from a tree?
%\end{itemize}
    %</note>
    \printbibliography
\end{document}
