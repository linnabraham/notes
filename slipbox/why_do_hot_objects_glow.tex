\documentclass{../template/texnote}

\title{Why Do Hot Objects Glow - Black Body Radiation}

\begin{document}
    \maketitle \currentdoc{note}
    %<*note>

URL: \url{https://www.youtube.com/watch?v=Mj2QOpQkSfI&t=3s}
\section{What is heat?}
The usual definition is that - \textcolor{teal}{Heat is the kinetic energy of atoms or molecules within matter}.
A more precise definition is to say that Heat is the excess energy that an atom or molecule has above it's ground state.
This can be thought of as coming from the changes in rotation, location or vibration.
\info{Is the segregation into potential and kinetic energy valid in this picture?}
These are all manifestations of excess energy.\info{Maybe that is why he is just using energy instead of kinetic}
\section{Moving to the Quantum world}
A change in energy of the electron would always produce a photon.
\section{Vibrations}
If a molecule vibrates with the right frequency, it can stimulate the excitation of an electron into a higher orbital and when the electron returns back to the ground state, a photon is released.\info{Is there a name for this stimulation process?}
Thus the mechanical energy transferred as heat to the material was absorbed and released back as electromagnetic energy of the photon.\info{So this means that atomic excitations are caused by changes in the vibrational levels of the atom or molecule?}
\info{Does this vibration require the presence of a dipole moment?}
A vibrating molecule may not always vibrate with the right frequency to excite one of it's electrons.
But when a collection of molecules are considered, the transfer of energy between them can lead to the excitation of electrons.
As we increase the temperature, more and more of these vibrations cause the excitation of more and more electrons.
Stronger vibrations can also cause the electrons to jump into even higher orbitals.
In matter there is overlap between nearby orbitals forming more orbitals that are accessible to the electrons.
So even though each orbital is associated with a specific energy (and thereby a specific frequency), because of the presence of large number of such orbitals there is a large variety of photon energies that these electrons can emit.\info{Is this why the spectra is a continous one?}
This is why the object appears white if it gets hot enough.\info{Why doesn't a continous spectra always produce white light?}
This is the basic mechanism for most blackbody radiation in the visible spectrum.
\section{Polar molecules}
When a molecule with a dipole moment undergoes rotations in the presence of an electromagnetic field, it can cause photons to be emitted as result of the interaction between the dipole moment and the field.
But since the forces involved here are so weak \unsure{Why are the forces very weak?} the energy of photon is also very weak (IR, microwave or radio?).
\section{Brehmstraulaung Emission}
\section{Why is it called Blackbody radiation?}
An ideal blackbody is one that absorbs all wavelengths that are incident on it.
Black is the colour we associate with a body that does this.
Therefore, blackbody radiation is that part of the radiation coming from an object that is not reflected or transmitted or in other words the radiation that comes specifically from the object itself.
    %</note>
    \printbibliography
\end{document}
