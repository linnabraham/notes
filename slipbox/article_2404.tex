\documentclass{../template/texnote}

\title{\textbf{Does the Universe Speak the Language of Radio Waves?}}[author={Linn Abraham}]

\begin{document}
    \maketitle \currentdoc{note}
    %<*note>
\section{Let there be Light}
Everyone is familiar with light and its sources, natural as well as artificial. The Sun, moon and the stars are some of the most well known natural sources of light. Bio-luminescence is also seen in nature in forms of fireflies etc. 
Fire was tamed by early men to be used as a source of both light and heat. For a very long time most of our lamps made use of fire. 
The electric bulb exploited the fact that substances glow when heated to very high temperatures.
Electricity provided a controlled way to heat certain materials to very high temperatures. It also provided a way to excite electrons in a gas and make them emit light on their own. 
Nowadays there exists more sophisticated ways to produce light such as LEDs that make use of the ability of semiconducting materials altered using a ``doping'' process, to emit light.

\section{An Electromagnetic Wave at Heart}
The great unification of electricity and magnetism took place when it came to be known that all magnetism is ultimately linked to electric fields and conversely that electricity can be generated just by changes in the magnetic field. Consequently it was realized that light was nothing more than an electromagnetic wave.
However this is only partly true. Because of the difference in their frequencies and hence energies, visible light and other forms of the electromagnetic spectrum manifest differently in their interaction with matter. 
Also the physical processes that lead to their generation are different. 
Combinedly this makes a world of difference and different observation windows in the electromagnetic spectrum sheds light on a different aspect or underlying phenomenon regarding the objects of study. 
Radio waves form a part of this electromagnetic spectrum, specifically the low frequency end of the spectrum.

\section{Radio Waves Arriving A Little Late to the Party}
After Maxwell put forward the theory of electromagnetism and predicted the existence of electromagnetic waves, Heinrich Hertz in 1887 became the first person to generate radio waves.
It is interesting to note that the unit of frequency itself is named after him. 
A new wireless form of communication was born to replace the telegraph and other sources of wired communication. It might be an understatement to say that this discovery of radio waves revolutionized communication. 
However because of their low frequecny and by extension low energy, radio waves have little to no effect on living organisms. In contrast even the adjacent regions of the spectrum like microwaves  and infrared has effects on living organisms. 
This is probably why the hand of evolution didn't find it worthy to endow any living organism with radio wave detectors similar to how the eyes and ears are used to detect visible light and sound waves. 
However imaginations of extra terrestrial life as portrayed in literature and films often depict aliens capable of directly communicating with each other through antenna like structures on their bodies.
It is because of this handicap that radio waves remained unknown for a long time and also why the need exists to convert radio waves to other forms like sound and light in order to enjoy content from our favourite radio stations.

\section{Fundamental Processes that Generate EM Waves}
\subsection{Acceleration of charges}

It is well known that the study of blackbody radiation lead eventually to the discovery of quantum mechanics and the quantum mechanical model of the atom. The Bohr quantization of orbits was put forward as an ad-hoc solution to a problem that could not be addressed in classical physics.
When trying to  know more about this problem, you come across the often repeated statement that electric charges emit radiation when they accelerate. Hence  electrons revolving around the nucleus of an atom cannot have stable orbits. The explanation for how this happens does not seem to be that simple but is beyond the scope of this article. If you take it as a given, a follow up question may be asked, is there a relation between the acceleration of the particle and the frequency of the emitted radiation? The answer seems to be that it is not a single frequency that is emitted but a range of frequencies. However there must be some relation between the nature of acceleration and nature of the distribution of frequencies. 
Coming back to the blackbody radiation, is acceleration of charges the fundamental processes that generates the blackbody spectrum? It must be, because blackbody radiation is fundamentally thermal in nature and a system of particles at a non-zero temperature undergoes vibrations, rotations and collisions etc. which is capable of producing electromagnetic radiations. %(Does it matter if the constituent particles are neutral?)

\subsection{Quantum state transitions}
The other well known way in which EM radiation is generated is due to transitions between different quantum mechanical states of the system. For e.g. the atomic emission lines in hydrogen and the hyperfine structure line in hydrogen at 21 cm wavelength.
These two mechanisms of EM wave generation are sometimes considered to be the thermal sources of radiation. Electromagnetic radiation can also be generated due to certain non-thermal sources which we shall now see in the context of radio wave generation. 

\section{Tuning into Radio Broadcasts from the Cosmos}
We saw that accelerating charges can produce radio waves along with radiation in other frequency ranges. And that the blackbody radiation is a manifestation of such charge accelerations at the microscopic level. 
%In most scenarios the acceleration produced is only sufficient to produce radiation that has a peak emission in the radio region. 
Radio waves are generated artificially by using time varying electric currents flowing in specially shaped metal conductors called antennae.  
Soon after the discovery of radio waves people started looking for natural sources of radio waves that are around us. The Sun is the biggest source of radio emission because of its high temperature and proximity. Thunderstorms are a source of radio noise since charged particles gets accelerated during these events. The field of radio astronomy opened up soon with the contributions of people like Karl Jansky, Grote Reber and Jocelyn Bell amongst others, who discovered several astronomical sources of radio waves. 
What other mechanism other than the blackbody radiation that we have already seen is capable of producing radio wave emissions?

\subsection{Continuum Emissions from Ionized Gas}
Plasma is the most common form of matter in the universe (99 percent of it). An ionized gas becomes a plasma when enough of the atoms are ionized and it exhibits collective behaviour. On Earth it can be found in the flash of a lightning and in auroras. Beyond the Earth’s atmosphere the Van Allen belts and the solar wind comprises of plasma. 
As mentioned before any body with a temperature above absolute zero emits blackbody radiation across a range of radiation including radio waves. However the region in which the bulk of the energy is emitted depends on the temperature of the body. For bulk emission in the radio wave region the temperature has to be less than 10 K which is true of dark dust clouds in the universe.
Thermal radiation has a characteristic that distinguishes it from non-thermal sources of radio waves. It produces a pure static hiss on a loudspeaker. 

\subsection{Non-thermal Sources of Radio Waves}
The non-thermal source of radio waves include the cyclotron, synchrotron and the astronomical masers. Such radiation arises as a result of the interaction of charged particles with magnetic fields. The fields makes it move in a circular or spiral path and the particle thus gets accelerated and radiates energy.
In contrast to thermal radiation  where the energy radiated increases with frequency. The intensity of non-thermal radiation usually decrease with frequency.

\section{How Messages Get Changed by the Medium}
The properties of the intervening media between the source and the detector also affects the observed radio spectrum. When oppositely charged ions recombine to a neutral state, the atom gets highly excited and several transitions occur. The resulting lines in emission or absorption are called recombination lines. Some of these lines particularly those due to carbon ions falls in the radio range of the spectrum. Another effect is due to the existence of quantized rotational states of molecules that fall in the microwave and long wavelength infrared regions of the spectrum. The spectral lines themselves can undergo doppler shifting due to the multiple reasons. This also affects the observed spectrum.

%\cite{wilson_tools_2009}

\begin{thebibliography}{6}
\providecommand{\natexlab}[1]{#1}
\providecommand{\url}[1]{\texttt{#1}}
\expandafter\ifx\csname urlstyle\endcsname\relax
  \providecommand{\doi}[1]{doi: #1}\else
  \providecommand{\doi}{doi: \begingroup \urlstyle{rm}\Url}\fi

\bibitem[1]{choudhuri_natures_2015}
Diane Fisher Miller.
\newblock \emph{Basics of Radio Astronomy for the Goldstone-Apple Valley Radio Telescope}.

\bibitem[2]{wilson_tools_2009}
Thomas~L. Wilson, Kristen Rohlfs, Susanne H{\"u}ttemeister, and Kristen Rohlfs.
\newblock \emph{Tools of Radio Astronomy}.
\newblock Astronomy and Astrophysics Library. Springer, New York, 5th ed
  edition, 2009.
\newblock ISBN 978-3-540-85122-6.

\bibitem[3]{wikipedia}
{Radio Wave}.
\newblock \url{https://en.wikipedia.org/wiki/Radio_wave}.

\bibitem[4]{stack01}
{How and why do accelerating charges radiate electromagnetic radiation?}.
\newblock \url{https://physics.stackexchange.com/questions/65339/how-and-why-do-accelerating-charges-radiate-electromagnetic-radiation}.

\bibitem[5]{stack02}
{What is the relation between the frequency of the light produced and the acceleration of the charged particle}.
\newblock \url{https://physics.stackexchange.com/questions/481679/what-is-the-relation-between-the-frequency-of-the-light-produced-and-the-acceler}.

\bibitem[6]{stack03}
{At the Fundamental Level, are Radio Waves and Visible Light Produced in the Same Way?
}.
\newblock \url{https://physics.stackexchange.com/questions/498709/at-the-fundamental-level-are-radio-waves-and-visible-light-produced-in-the-same?rq=1}.
\end{thebibliography}

\vspace{0.5cm}

\noindent\fbox{%
	\parbox{\textwidth}{%
		\textbf{About the Author}\vspace{0.2cm} \\
		\textbf{Linn Abraham} is a researcher in Physics, specializing in A.I. applications to astronomy. 
He is currently involved in the development of CNN based Computer Vision tools for
prediction of solar flares from images of the Sun, morphological classifications of galaxies from optical images surveys and radio galaxy source extraction from radio observations.
	}
}
    %</note>
    \printbibliography
\end{document}
