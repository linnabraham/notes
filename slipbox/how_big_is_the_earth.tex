\documentclass{../template/texnote}

\title{How Big Is The Earth?}

\begin{document}
    \maketitle \currentdoc{note}
    %<*note>
One of the best arguments for the round-earther's claim of a spherical Earth is that the shape of the Earth's shadow on the moon is circular during any lunar eclipse.
During a lunar eclipse, the Sun illuminates the Earth from behind and its shadow fals on the moon.
Regardless of when or where this happens, the shadow is always circular and a sphere is the only shape that fits such a scenario.
This fact was known during the time of Erastothenes, who lived a very long time ago and which is evident from the name alone.
We know that the Sun is so far away that the light rays coming from it are parallel.
Erastothenes also assumed this to be true and went about calculating the Earth's radius based on this assumption.
He found out that at the same time when the Sun is directly overhead at Syene in Egypt, the Sun cast a shadow that is 7 degrees inclined from the vertical at Alexandria.
By knowing the actual distance from Syene to Alexandria and the angular distance of 7 degrees, the circumference of the Earth can be calculated.
The final answer is 6371 km.

    %</note>
    \printbibliography
\end{document}
