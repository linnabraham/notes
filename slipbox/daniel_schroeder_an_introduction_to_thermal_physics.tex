\documentclass{../template/texnote}

\title{Daniel Schroeder An Introduction To Thermal Physics}

\begin{document}
    \maketitle \currentdoc{note}
    %<*note>

\section{\citetitlebyauthor{schroeder_introduction_2021}}

\subsection{Alternative definition for temperature}
Temperature is the thing that’s the same for two objects, after they’ve  been in contact long enough.

But this definition  is extremely vague: What kind of ``contact'' are we talking about here? How long is  ``long enough''? How do we actually ascribe a numerical value to the temperature?  And what if there is more than one quantity that ends up being the same for both objects?

\subsection{Thermal Equilibrium}
After two objects have been in contact long enough, we say that they are in  thermal equilibrium.  The time required for a system to come to thermal equilibrium is called the  relaxation time.

For each type of equilibrium between  two systems, there is a quantity that can be exchanged between the systems:
%Exchanged quantity Type of equilibrium  energy thermal  volume mechanical  particles di⇧usive

Thermal - Energy

Mechanical - Volume

Diffusive - Particle

Notice that for thermal equilibrium I’m claiming that the exchanged quantity is  energy.

    %</note>
    \printbibliography
\end{document}
