\documentclass{../template/texnote}

\title{Notes On Solar Flares}

\begin{document}
    \maketitle \currentdoc{note}
    %<*note>
	\subsection{What are sunspots? What is their relevance to the study of the Sun?}

The Sunspots showed that the Sun has a differential rotation.
The sunspots are seen to have a periodicity in their appearance.
The differential rotation of the sun keeps distorting the magnetic field of the sun.
Sunspots are basically patches of magnetic field concentration sitting in a region of the sun where vigorous convection is going on.

\subsection{What is a magnetogram?}

Refer to notes on Arnab’s book in Zotero



\subsection{Why is there a convection zone in the outer layers of the Sun?}

Refer to notes on Arnab’s book in Zotero

\subsection{What are the different layers of the Sun?}

The core
The radiative zone
The convective zone
Photosphere
Chromosphere
Corona

\subsection{What is the Sun and what does it do?}

A massive ball of hydrogen becomes dense due to its gravity. Gas turns into plasma. Conditions are right to start the nuclear fusion process (p-p chain) which creates energy (converts binding energy) in the form of gamma ray photons. 

99 percent of the Sun’s energy is produced in its core. The high energy gamma rays produced in the core drop down to the visible region by the time it reaches the photosphere. This is because of the constant absorption and re-emissions suffered by the photons as they travel through the radiative and convective zones. Can these be called collisions? What about the reverse compton effect? Where is the energy of these photons going then? To the KE of the particles?

\subsection{The Sun and Its Magnetism}
Sunspots were seen to be regions of high magnetic field after the discovery of Zeeman effect which allowed us to do that. Since matter in the Sun is in the form of plasma it is natural that it has a magnetic field. Would it have a “net magnetic field” is called the question of the self-excited fluid dynamo in MHD. There are still gaps in our ability to use the equations of MHD to fully explain the 11 year solar (magnetic) cycle.


\subsection{Known Unknowns:}
\begin{itemize}
	\item Solar Energetic Particles (SEP), Plumes.
	\item Active Regions and Sunspots
	\item CME and Halo CME
	\item Filaments and Prominences
	\item Homologous flares

\end{itemize}
CME involve particles so take more time to reach the earth. Whereas for flares, we do not get advance warning. This is why we need to predict them. 

\begin{itemize}
	\item Understanding more about the data?
	\item Does the LOS data depend on the particular observatory and its location?
	\item What about the vector magnetograms? How are they computed? 
	\item Average mag field on photosphere
	\item Observing the Sun
	\item What are we viewing when we view the Sun in different filters?
\end{itemize}

\begin{itemize}
\item Why do sunspots appear dark? Because of flux freezing. Because of 
\item Existence of current sheets between anti alligned magnetic field lines.
\item Software to watch the Sun
\item L1 point and why does Aditya orbit around an empty point?
\item Thermal and non thermal radiation?
\item The solar constant
\item The energy falling per unit time per unit area perpendicularly outside the earth’s atmosphere.
\item Or Solar power falling perpendicularly on unit area.
\item What does the effective temperature of the sun or any star actually mean wrt to the black body curve?

\item Limb darkening

\end{itemize}

\subsection{Article on the Sun}
\begin{itemize}
\item Blackbody curve of the Sun
\item Origin of sunspots
\item Magnetic field of Sun
\item State of matter in the Sun
\item Structure and atmosphere of the Sun
\item Opacity of the Sun
\item How much of a star or the Sun can we actually see?
\item Compare this with the visible universe of the early epochs of the universe
\item The corona heating problem
\item Question of how is the Sun powered
\item Contradiction between geological evidence and kelvin helmotz ideas
\item Solar neutrino problem

\end{itemize}












\subsection{What do we know about the Sun and how do we know it? }
\begin{itemize}
\item People must have seen the corona of the Sun during eclipses.
\item People came to know about spots on the Sun’s surface and studied them in detail after the discovery of the telescopes.
\end{itemize}


    %</note>
    \printbibliography
\end{document}
