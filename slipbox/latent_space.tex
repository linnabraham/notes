\documentclass{../template/texnote}

\title{Latent Space}

\begin{document}
    \maketitle \currentdoc{note}
    %<*note>
    There can be a confusion between latent manifolds, as in the manifold hypothesis \parencite[][Chapter 5.1.2]{cholletDeepLearningPython2025} and latent spaces.
    Lets understand what is meant by latent manifold first.

    Consider the MNIST problem in which the input to the neural network is an array of integers - 28 x 28 in size with each integer being between 0 and 255.
    The total number of possible inputs is 256 to the power of 784.
    According to the manifold hypothesis, all natural data, are not spread randomly or uniformly in this space\improve{Keep in mind that the raw input space is an integer grid initially but often replaced by a continuous space by normalizing the values to floats between 0 and 1}\unsure{But even floats when represented digitally are discrete right?}
\info{The range and nature of the possible values might affect the scale and shape of this space}
    but fall on a subspace that can be called the latent manifold.
    What this means with reference to the MNIST example is that, any random point in the space we mentioned before would not correspond to a realistic handwritten digit.
    The goal of a deep learning model would be to learn the structure of this low-dimensional space using far less parameters\unsure{Is this true or are there cases where models might have far more parameters than in the original space?}.
    
    The neurons in the hidden layers of a neural network or feature maps in a CNN are intermediate representations of the data.
    This intermediate space which might have a different dimensionality and structure could be called the latent space.
    \unsure{Does the manifold hypothesis still hold in this space?}
    In the case of a classification task, we can imagine this space to be more beneficial to the network, say because data points corresponding to different classes could be well separated here.
    %</note>
    \printbibliography
\end{document}
