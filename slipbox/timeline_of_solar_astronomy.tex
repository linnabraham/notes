\documentclass{../template/texnote}

\title{Timeline Of Solar Astronomy}

\begin{document}
    \maketitle \currentdoc{note}
    %<*note>
	\subsection{Timeline}
	\citetitlebyauthor{choudhuriNaturesThirdCycle2015}
	\begin{itemize}
\item 1600: Gilbert proposed that the earth is a magnet. 
\item 1610: Galileo began his study of sunspots using the telescope. 
\item 1844: Schwabe discovered the sunspot cycle. 
\item 1859: Carrington observed a solar flare (the first such observation by anybody) and noted the geomagnetic storm several hours later.
\item 1892: Kelvin ‘proved’ that solar disturbances cannot affect the earth. 
\item 1908: Hale discovered that sunspots have strong magnetic fields. 
\item 1933: Cowling proved the anti-dynamo theorem. 
\item 1955: Parker formulated the dynamo equation and proposed the first theoretical model of the sunspot cycle. 
\item 1958: Parker developed the theory of the solar wind, which was discovered a few years later.
	\end{itemize}
	\subsection{Timeline}
	\citetitlebyauthor{priestMagnetohydrodynamicsSun}
\begin{itemize}
	\item Eclipses are one of the awe inspiring things that ancient people must have observed about the Sun.
	\item Since it is not possible to directly observe the Sun, techniques that enable projection of the image of the Sun need to predate it.
	\item 165 BC - The Chinese have observations of sunspots
	\item 1609 - Kepler suggests that the Sun might have a magnetic field to keep the planets in orbit around it.
	\item 1610 - Sunspots are observed using the recently invented telescope by many including Galileo Galilei
	\item 1633 - Galileo is sentenced to house arrest for his heretical heliocentric views.
	\item 1645–1715 - Period of very few sunspots (Maunder Minimum)
	\item 1666 - Newton formulates his law of gravitation and applies it to the Sun and the planets.
	\item 1814 - Fraunhofer observes the lines in the solar spectrum.
	\item 1842 - Prominences, chromosphere and corona are observed clearly during a solar eclipse
	\item 1843 - Schwabe notices the 11 year sunspot cycle.
	\item 1851 - Corona photographed for first time during an eclipse.
	\item 1852 - An association between the sunspot cycle and geomagnetic storms is found.
	\item 1858 - Carrington discovers that the sunspots move to towards lower and lower latitudes with progress in the solar cycle.
	\item 1859 - Carrington observes a solar flare possibly for the first time.
	\item 1868 - During an eclipse a new element named Helium is observed in the Sun.
	\item 1874 - A detailed description of the fine structure of the photosphere called granulation is given.
	\item 1877 - Spicules are described.
	\item 1889 - Hale invents spectroheliograph
	\item 1908 - Hale discovers the presence of strong magnetic field in the sunspots.
	\item 1919 - Hale and Joy discovers that pairs of sunspots have opposite magnetic fields.
	\item 1920s - Dominance of hydrogen and helium in the atmosphere and interior of the Sun is realized.
	\item 1930 - Lyot invents the coronograph to view the corona without an eclipse.
	\item 1934 - Cowling proposes a theory for sunspots and an anti-dynamo theorem.
	\item 1938 - The C-N and P-P chains are proposed by Bethe as an explanation for the source of the Sun’s energy.
	\item 1939 - The high temperature $>10^6$ K prevalent in the Corona is realized.
	\item 1942 - Radio emissions from the Sun are detected by radar.
	\item 1948 - It is proposed that the outer atmosphere is heated by sound waves propagating up from the convection zone.
	\item 1952 - The magnetograph is invented and used to discover properties of the photospheric magnetic fields.
	\item 1955 - Parker produces major work on the dynamo and magnetic buoyancy theories.
	\item 1956 - The basic theory of MHD is summarised by Cowling in his book.
	\item 1957 - Satellite observes interplanetary plasma.
	\item 1958 - Observations of reversal of polar fields at sunspot maximum.
	\item 1958 - Parker predicts the solar wind and proposes a model.
	\item 1960 - Five minute oscillations in the photosphere are discovered.
	\item 1972 - Tousey and Koomen observe a CME from OSO 5.
	\item 1973 - Skylab explores the corona in detail in soft X-rays with its holes, loops and X-ray bright points.
	\item 1980 - Hickey et. al. discover that the solar irradiance is varying, solar constant is not constant.
	\item 1980s - Important advances in MHD theory in areas of equilibria, waves, instabilities and reconnection.
	\item 1990s - Yohkoh revelas the dynamic nature of the corona and the presence of magnetic reconnection in solar flares.
	\item           - SOHO and LASCO produces a major increase in understanding.
	\item 1998 - TRACE provides high resolution view of the corona.
	\item 2002 - RHESSI transforms our knowledge of solar flares.
	\item 2000s - STEREO views CME in three dimensions.
	\item  HINODE studies link between photosphere and corona and changes our paradigm for photospheric magnetic fields.
	\item 2010 - SDO begins a new revolution in understanding with super-TRACE and super-MDI instruments.
\end{itemize}


    %</note>
    \printbibliography
\end{document}
