\documentclass{../template/texnote}

\title{How Far Away Is The Sun}

\begin{document}
    \maketitle \currentdoc{note}
    %<*note>
From Kepler's third law of planetary motion, we know that 
$$ T^2 \propto a^3$$
Where T is the orbital period of a planet around the Sun and a is the semi-major axis of the orbit.
By taking ratios one can convert this to an equation.
$$ \frac{T_1^2}{T_2^2} = \frac{a_1^3}{a_2^3} $$
Now let us put Earth as the second planet and Venus as the first planet. We define the Earth's orbital period around the Sun as one year and its semi-major axis as one astronomical unit (AU).
This allows us to calculate the distance from Venus to Sun in terms of AU if we know the orbital time period of Venus. $$ T_1^2 = a_1^3$$
For the purposes of this discussion, let me call the semi-major axis length itself as the distance.
Since the orbital period of Venus can be found by observations, we can find the distance between Venus and Sun in terms of AU.
One minus this distance would then give the distance between Earth and Venus in terms of AU.
Now consider what happens during the Venus transit, that is, when the Venus comes directly between the Sun and Earth.
If at this time, we measure the actual distance from Earth to Venus, we can find out the length of one AU.
Such a distance measurement can be done using the parallax method.
Doing all the calculations we arrive at an answer of 150 million kilometers for the length of one AU.

    %</note>
    \printbibliography
\end{document}
