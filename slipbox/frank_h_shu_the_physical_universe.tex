\documentclass{../template/texnote}

\title{Frank H Shu The Physical Universe}

\begin{document}
    \maketitle \currentdoc{note}
    %<*note>

\section{\citetitlebyauthor{shu_physical_1982}}
Refer to Chapter 4 of \citetitlebyauthor{shu_physical_1982}
\subsection{Laws of thermodynamics}

Zeroth law: Heat always diffuses from hot to cold, with the temperature\unsure{Is temperature defined using this law, i.e. as the property that becomes uniform when TE is reached?} becoming uniform when thermodynamic equilibrium is reached.\info{If so, one definition of temperature would be - (Definition 1) Temperature is that property which becomes same between two objects when they reach equilibrium together?}

The first law is trivial to understand, it is just a restatement of the law of conservation of energy.
First law: Heat is a form of energy. When this is taken into account, energy is always conserved.

Second law: Some forms of transformations of one kind of energy to another do not occur in natural processes.
The allowed transformations in a closed system are always characterized by a nondecreasing entropy (introduction of more macroscopic disorder).
In open systems where the entropy may be kept constant\unsure{Why might entropy be constant in an open system?}, the allowed transformations are always characterized by a decrease in the amount of (free) energy\unsure{Is free energy defined as the energy available to do useful work?} available to do useful work.

%\vspace{2 \baselineskip}
\subsection{Comments on Second law of Thermodynamics}

..the fundamental irreversibility of most natural processes on the macroscopic level (the concern of all ecologists) seems, at first sight, to be at odds with the fundamental time-reversibility of the microscopic laws of physics. How is this apparent paradox resolved?

\citeauthor{shu_physical_1982} explains this using the example of a billiards game.
He goes on to say this - 
This statistical improbability is what reconciles the strong adjectives ``always'' and ``never'' in the usual statements of the second law of thermodynamics with the fundamental time-reversibility of physics at a microscopic level.

    %</note>
    \printbibliography
\end{document}
