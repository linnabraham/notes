\documentclass{../template/texnote}

\title{Linux Cli}

\begin{document}
    \maketitle \currentdoc{note}
    %<*note>
Lecture 5: Command-line environment 2020 (\url{https://youtu.be/e8BO_dYxk5c})
\section{Job control}
	\begin{itemize}
		\item \& - program starts but doesn’t take over the prompt
		\item \texttt{jobs} - see running jobs
		\item \texttt{bg}   - to continue a suspended job where 1 is the job id.  
		\item \texttt{kill -STOP}  - suspend a job using job id

		\item \texttt{kill -HUP} - to hang up a job. 
		\item If you start a  job with \texttt{nohup} like \texttt{nohup sleep 2000} then even if a HUP signal is passed nothing happens.
		\item \texttt{kill -KILL} - kills a job no matter what
		\item \texttt{fg} to re-attach a background job to foreground

	\end{itemize}


\section{Tmux}
	\begin{itemize}
		\item Hierarchy
		\item Sessions have windows
		\item windows have panes
		\item Windows - Tabs in other places
		\item  `Prefix + 1' jumps to the window 1 in a session
		\item  `,' (comma) can be used to rename window
		\item `z' - zoom into a pane, repeat to undo
	\end{itemize}
    %</note>
    \printbibliography
\end{document}
