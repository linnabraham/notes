\documentclass{../template/texnote}

\title{Topical Course Lecture Last}

\begin{document}
    \maketitle \currentdoc{note}
    %<*note>
Radiative model.
dp/ dr - hydrostatic equilibrium
dM / dr -  definition of mass distribution
dR/dr nuclear energy generation comes from energy balance - whatever is generated should be radiated away.
dT/dr Radiative diffusion. - how generated energy gets through and out of the star.
Temperature gradient equation came from momentum balance between matter and radiation.


Pressure P (density, T) = sum of gas pressure and radiation pressure.
Mass density and Temperature are taken to be the two thermodynamic variables.

Nuclear energy generation epsilon ? 
Rosseland opacity - All absorption and scattering process, add up all opactities and take inverse of it.
Mean of mean free paths 

4 equations and 3 consecutive relationships.
rho, T, L , M - unknown funs of r
Boundary condition 
At the center, luminosity is 0. - no source of luminosity.
Mass is 0.
rho and T at outer boundary is 0.
Automatically we see the pressure must be zero, from above equation.

Reff is some effective radius.
There must be some surface acting as a blackbody.

4pi Reff2 times sigma times T4(Reff) = L = script L(Reff)
Integral over Reff to inf rho of r times kappa (r) approx to 1
Optical depth to Reff approx unity
The script M is many times a better variable than r.

Mass for a given chemical composition - open cluster pleideas (same chemical composition) a thin HR diagram
in globular cluster - there is variation in chemical composition and age.
With age, the chemical composition changes.
The original chem composition of the Sun can still be seen in the outer layers of the Sun.
Most important qty is the mass of the star - Vogt-Russell theorem.


Opacity
Rossleand, freq integrated - absorption and scattering
scattering - thomson scattering due to free electron
absorption -
free - free = brehmstraulaung
bound - free 
bound - bound

A free electron cannot absorb a photon but it can scatter. elastic? 
The wall doesnt absorb much energy but absorbs the momentum.
Analogous to a tennis ball hitting on a wall.


Bound free - there is significant contribution from outer orbitals which is almost same as free free.

Thomson scattering cross section is indepdene of nu which neednt be the case.
Hence the roseland thing would be independent of frequ.

Free - free absorption

The deflection would leave to velocity v same as initial. - Coulomb scattering 
in general both ion and electron are moving. 
We can take the ions frame.


The reduced mass is almost same as electron mass - why?

The ratio of emission coefficient to absorption coefficient is the source function which is equal to the plank dist.

For dealing with absorption it is easier to use classical approach whereas using QM we can directly do the emission.

From Parseval's theorem the spread of frequency must be 1 over t where t is the significant time in our calculation.

We can assume that the distribution of energy over the range of frequencies is the same.

Find the rate of encounters of one ion with all electrons with e in v, vplusdv and b,bplusdb

    %</note>
    \printbibliography
\end{document}
