\documentclass{../template/texnote}

\title{Notes On Meghnad Saha Thermodynamics}

\begin{document}
    \maketitle \currentdoc{note}
    %<*note>

\section{\citetitlebyauthor{saha_treatise_1969}}
\subsection{Thermal Equilibrium}
When two bodies are brought in contact or communicate with
each other through a wall, it is found that, in general, there is a
change in their properties, such as volume, pressure, etc., due to
exchange of heat. Finally a state is attained after which there is
no further change as long as external conditions do not change.
This is called an equilibrium state of the combined system. The
two bodies are then said to be in thermal equilibrium with each other.

\subsection{Definition of Temperature}
Experimentally it is seen that,
Two systems in thermal equilibrium with a third are in thermal
equilibrium with each other.
This law of thermal equilibrium is the basis of the existence of
the concept of temperature.

\ldots All these three bodies may be said to possess a
property that ensures their being in thermal equilibrium with one
another. We call this property temperature. The temperature of a
system is a property which determines whether or not a system is
in thermal equilibrium with other systems.

%Temperature is defined as follows in \citetitlebyauthor{m._w._zemansky_heat_1996},
%The temperature of a system is a property that determines whether or not a system is in thermal equilibrium with other systems.

\subsection{Another definition of thermal equilibrium}
In chapter 5, he gives a much more compact definition of thermal equilibrium, this time using temperature.
If the temperature in all parts of a system is uniform and the
same as that of the surroundings, the system is said to be in thermal
equilibrium.\info{But this has the potential of being a circular definition if taken standalone.}\unsure{What is Local Thermodynamic Equilibrium?}

    %</note>
    \printbibliography
\end{document}
