\documentclass{../template/texnote}

\title{Bowley Sanchez Chapter 8}

\begin{document}
    \maketitle \currentdoc{note}
    %<*note>

\citetitlebyauthor{bowley_introductory_1996}\\
A black body is defined as one which absorbs all the radiation which is
incident upon it, none is reflected. When it is cold it looks black \info{Good to point out that it is only black when it is cold}: any radiation
which is incident on the body is absorbed. The same body, when heated, will
glow. If you were to look through a tiny hole in a furnace, then the reddish glow
you would see would be a fair approximation to the visible portion of black-
body radiation.
..
What you see is only a small part of the total
spectrum of radiation.


All objects emit electromagnetic radiation at a rate which varies with their
temperature.
When a body is in thermal equilibrium with its surroundings, it emits and
absorbs radiation at exactly the same rate. If it emits radiation poorly it must
absorb radiation poorly. A good absorber is a good emitter. The ability of a
body to emit radiation is directly proportional to its ability to absorb radiation.
When the temperature of a body exceeds that of its surroundings, it emits more
radiation than it absorbs.

There is much less radiation from the tungsten filament in a light-bulb than
from a black body at the same temperature.
The spectrum for the tungsten
filament is also shifted slightly to shorter wavelengths.
These differences arise
because the tungsten filament does not absorb all radiation incident on it, so it
emits less than a black body.


It is convenient to
discuss black bodies in terms of an idealized model which is independent of the
nature of any material substance.\info{See! This is where the problems start}


The best model is a large cavity with a very
small hole in it which absorbs all the incident radiation. The radiation hits the
walls and rattles around inside and comes to thermal equilibrium with the wall:
before emerging from the hole. The smaller the hole, the longer it takes before the radiation emerges.
Kirchhoff (1860) gave a good operational definition: Given
a space enclosed by bodies of equal temperature, through which no radiation can
penetrate, then every bundle of radiation within this space is constituted, with
respect to quality and intensity, as if it came from a completely black body at
the same temperature.

Suppose that the walls of an oven surrounding a cavity are heated up to a
temperature T. It is found experimentally that the radiation from the hole in
the cavity depends only on the temperature of the oven walls and on the area,
.4, of the hole. The radiation is independent of the material that makes up the
walls.

The Stefan-Boltzmann law tells us how much energy is emitted per second,
but tells us nothing about how the radiated energy is divided into the infinite
number of wavelengths that are present in black-body radiation.
\\

{\Large \textbf{The Rayleigh-Jeans theory} }\\


The black body consists of electromagnetic radiation in thermal equilibrium with
the walls of the cavity. When they are in thermal equilibrium, the average rate
of emission of radiation by the walls equals their average rate of absorption of
radiation. The condition for thermal equilibrium is that the temperature of the
walls is equal to the temperature of the radiation. But what do we mean by the
idea that radiation has a temperature?

We could imagine that the material of the walls contains charged particles
which oscillate about their equilibrium positions. (It does not matter what the
walls are made of, so we can mentally imagine them to consist of oscillating
charges.) 

As was known from Maxwell's work on electromagnetism, a 
moving charge \unsure{Is it moving or accelerating charge?} radiates an electromagnetic wave. We imagine the walls to be full
of charged particles jiggling about, coupled to electromagnetic standing waves.
Suppose each standing-wave mode of the electromagnetic field is coupled to an
oscillator in the wall which oscillates with the same frequency as the standing
wave.\unsure{Snehil asks this, for electrons on the walls if they execute other oscillations that results in radiation that goes away from the cavity towards the outside 
do they have to satisfy this condition?}
Each oscillator has two degrees of freedom, one for the kinetic energy, one
for the potential energy, and so it has an average energy of kT according to the
equipartition theorem.\unsure{Snehil asks, why doesn't the kinetic energy term itself have three degrees of freedom? - Is it because the oscillators are thought to be constrained such that they can only vibrate in one dimension?}
In thermal equilibrium the average energy of the 
oscillator and the average energy of the standing-wave mode of the electromagnetic
field must be the same for the two to be in thermal equilibrium. Hence each
mode of oscillation of the electromagnetic field has an energy kT and can be
thought of as having a temperature T. This is the basis of the Rayleigh-Jeans
theory.

However, there is a second respect in which the Rayleigh-Jeans formula is correct. It is now known to work perfectly lor long wavelengths thanks to Jeans (and
Einstein) getting the numbers correct.  This partial agreement of the calculation
with experiment showed that the theory is not totally wrong, but something is
missing. The question is this: which classical law is wrong for small wavelengths? \info{Good question}

Planck decided that he would not assume that the average energy of an oscillator in the wall was equal to kT.

Planck's formula fits the experimental data amazingly
well. He wanted to understand why this was so. \info{Read this rest. This looks to be an interesting take.}

Planck concentrated his attention on
the statistical mechanics of the oscillators in the wall. He treated the oscillators
in the wall as if they were thermally isolated from the rest of the universe so he
could use the techniques of the microcanonical ensemble.

Planck wanted to
find a way of obtaining eqn (8.3.9) for the entropy by calculating W directly and
using the relation S = k ln(W). 
This is how he did it.


Consider a large number N of oscillators all with frequency v. The total 
energy is $U_N$ =  NU and the total entropy $S_N$ = NS = k ln(W) where W is the
number of arrangements for distributing the energy $U_N$ amongst the N 
oscillators. Planck imagined that the total energy is made up of finite energy elements
of size e, so that $U_N$ = $M e$ where M is a large integer. This is Planck's quantum
hypothesis. W is taken to be the number of ways in which M indistinguishable
energy elements (quanta) \unsure{From discussion with Snehil, maybe he is not talking about the quanta of the em field by the quanta of the energy of the charges. Also is he talking about electrons or atom as a whole since electrons are not distinguishable particles or are they? since they can be distinguished by position and momentum.}
can be arranged amongst N distinguishable oscillators.\unsure{Here radiation is treated as indistinguishable and electrons as distinguishable. Why?}


Planck introduces two new ideas: not only is energy quantized, but the 
counting involves indistinguishable energy elements or quanta. This was an entirely
new way of counting the number of arrangements of the energy amongst the
oscillators.


Planck quantized the oscillators which he imagined to exist in the wall of the
cavity. He was unaware that his quantization proposal could be applied to 
oscillations of the classical radiation field itself.
%\newpage

The molecular composition of the atmosphere of a star or a planet depends on
the escape velocity.\info{Move this to it's own section}
\unsure{How can this be applied to the Sun's atmosphere?}

\section*{Surface of a black body?}
The usual model of a blackbody is a cavity with a hole.
What difference arises if there are, in addition, particles inside the cavity?
Also the source of heat is external to the blackbody itself.
What difference would arise if the source of heat is also inside the blackbody?
As is the case in the Sun and other stars?

    %</note>
    \printbibliography
\end{document}
