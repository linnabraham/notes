\documentclass{../template/texnote}

\title{What Is A Black body (Stefan Boltz-
mann’s Law, emissivity, grey and white
bodies...) - Physics}

\begin{document}
    \maketitle \currentdoc{note}
    %<*note>
URL: \url{https://www.youtube.com/watch?v=EWqzg0iqNtw}
\\

1:50 mins - What does it mean to say that light is absorbed? 
When light interacts with the surface\unsure{Is this usage deliberate?} of a body, because of the charge contained in electrons (and protons) they start to oscillate \unsure{Does an electron in a stable orbital in an atom oscillate? or are we talking here about free electrons?} - they gain kinetic energy \unsure{Were the electrons not oscillating prior to their interaction with light? If not then that means that they go into a higher vibrational state than before, if yes, then they are already emitting radiation of their own?}
So the energy carried by the light has been transferred to the KE of the charged particles of the body.
%Since the temperature is propotional to the average kinetic energy of the electrons, as a result of the increase of kinetic energy, the temperature increases.
How the black body is heated may differ between cases. In some cases the heating might be due to incident electromagnetic radiation and in other cases like in stars the heating comes from nuclear fusion.
\unsure{So inspite of the different ways in which the object may be heated, the emission spectrum is still the same ?}


    %</note>
    \printbibliography
\end{document}
