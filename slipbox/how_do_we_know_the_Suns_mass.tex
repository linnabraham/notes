\documentclass{../template/texnote}

\title{How Do We Know The Sun's Mass}

\begin{document}
    \maketitle \currentdoc{note}
    %<*note>
Henry Cavendish (1731-1810) is the man credited with weighing the Earth for the first time.
He did this by measuring the gravitation constant that appears in Newton's Law of Universal Gravitation.
He was able to do this because the radius of the Earth was already \exhyperref{HowBigIsTheEarth}{known by that time}.
Knowing the Earth's mass makes us one step closer to finding the mass of the Sun.
We know that the Earth goes around the Sun in a sort of circular motion. The centripetal acceleration is provided by the force of the Sun's gravitational attraction on the Earth.
The centripetal acceleration in uniform circular motin is given by $\frac{v^2}{r}$. Where v is the speed of Earth's motion around the Sun. The speed of Earth's motion around Sun is trivial to find once we know the circumference of the Earth's orbit. We divide that quantity by the time taken for one full revolution which is one year.
By equating this to the force of gravitation between Sun and Earth, we can find out the mass of the Sun since we already know the mass of the Earth and the \exhyperref{HowFarAwayIsTheSun}{Sun-Earth distance}.

    %</note>
    \printbibliography
\end{document}
