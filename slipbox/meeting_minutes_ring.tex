\documentclass{../template/texnote}

\title{Meeting Minutes Ring}

\begin{document}
    \maketitle \currentdoc{note}
    %<*note>
        %\section{Meeting minutes}
	\section{May 19th, 2025}
		\begin{itemize}
			\item Discussion with Arif Babul
			\item Since we have a set of pristine images (simulated) on which various filters (SDSS bandpass or even blurring filters) can be applied, could this be used for training?
			\item Because simulations allow you to go back in history you can see when a ring appears and might probably disappear too.
			\item If you can get the same galaxy across redshifts that could also be valuable information to train a network?
			\item Can we use the sdss and subaru counterparts of simulated galaxies to train or re-train (a model already trained on SDSS) a model which enables us to find rings in Subaru using a trained model with SDSS.
			\item We could take a set of galaxies and then take the same set at greater redshift, visually label rings and non-rings from the low redshift set and then test the performance on the higher redshift one.

		\end{itemize}
    %</note>
    \printbibliography
\end{document}
