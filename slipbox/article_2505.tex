\documentclass{../template/texnote}

\title{\textbf{The Hunt for the Cosmic Radio Lines of Neutral Hydrogen}}[author={Linn Abraham}]

\begin{document}
    \maketitle \currentdoc{note}
    %<*note>

The universe communicates with us not just through the visible light our eyes can see, but across the entire electromagnetic spectrum.
The dawn of radio astronomy opened an entirely new window onto the cosmos, beginning with pioneers like \textbf{Karl Jansky}, who accidentally discovered a ``steady hiss type static of unknown origin'' in 1932 while studying interference on transatlantic radio telephone service.
Jansky's work revealed radio waves apparently coming from space, particularly from the plane of the Milky Way.
Despite his recognition of the importance of this discovery, his work was not immediately appreciated by radio engineers, astronomers, or physicists.

Following Jansky, individuals like \textbf{Grote Reber} took up the challenge, building dedicated radio telescopes in his backyard.
Reber produced the first radio contour maps of the sky, beautifully outlining the Milky Way and hints of discrete sources.
His work helped demonstrate the potential of studying the cosmos at radio wavelengths.
Early theoretical work sought to explain the observed galactic background radiation, considering mechanisms like thermal free-free emission from interstellar gas.
However, calculations showed that Jansky's observed brightness temperatures of around 100,000 K at low frequencies could not be explained by thermal emission from gas at typical interstellar temperatures of $\sim$10,000 K.
This discrepancy pointed towards non-thermal processes, leading later to the suggestion of the synchrotron mechanism, involving cosmic rays in galactic magnetic fields.

Building on these foundational observations and techniques, radio astronomers sought to extend spectroscopic analysis -- the study of specific wavelengths or frequencies -- into the radio domain, much like optical astronomers used it to determine stellar properties.
\textbf{A crucial goal in this quest was the search for a single, profoundly important cosmic radio signal: the 21 cm line of atomic hydrogen}

\section{The Bold Prediction (Amidst Darkness)}

The intellectual journey toward discovering this specific radio line began in a surprisingly challenging environment: Nazi-occupied Holland during World War II.
Amidst these difficult circumstances, a student named \textbf{Hendrik van de Hulst} at Leiden chose to continue his studies after his professor was detained.
Working with J.H. Oort, who was impressed by the potential of radio waves for studying the galaxy's structure (having learned about Reber's reports from a smuggled journal), Van de Hulst tackled the theoretical problem of what specific radio signals interstellar hydrogen might emit.

Van de Hulst's theoretical work is described as ``remarkable both for its scientific prescience and for the conditions under which it was produced''.
While he considered other possibilities, his most significant contribution was the first discussion of the implications of the \textbf{21 cm hyperfine line of atomic hydrogen}.
This line arises from a specific, low-energy transition within the neutral hydrogen atom.
Van de Hulst calculated the wavelength to be \textbf{``21.1 cm, corresponding to a frequency of 1420.4 MHz''}.
He noted that receiver sensitivity would need to improve significantly for detection, acknowledging the speculative nature of this transition but recognizing its immense potential given hydrogen's abundance and the ideal wavelength for interstellar penetration.

\section{Independent Confirmation and Other Ideas}

Working independently in the Soviet Union, \textbf{Iosif Shklovsky} was also motivated to predict cosmic radio lines after seeing a brief mention of Van de Hulst's ideas.
Even without access to the original article, Shklovsky performed his own calculations concerning the 21 cm line in a ``largely independent manner''.

Shklovsky's paper was the first to \textbf{calculate the probability for the hyperfine transition}, achieving a result ``off by only a factor 4'' from the correct value.
He argued for collisional excitation of the line and demonstrated the ``feasibility of detection of the line with available equipment''.

Remarkably, Shklovsky's work also included an ``entirely original second portion'' concerning potential radio line radiation from deuterium and, significantly, \textbf{interstellar molecules} like OH and CH.
He was ``enthusiastic about the odds for detection'' of these molecular transitions.
This part of his work, anticipating the ``rich field of galactic microwave spectroscopy'' by two decades, wasn't widely followed up for several years.

\section{The Breakthrough Discovery}

Despite these theoretical predictions and encouraging calculations, the 21 cm line remained undetected for several years due to the ``difficulties of receiver technology''.
When the discovery finally happened in 1951, it occurred ``as so often happens in modern science,'' with detection and confirmation achieved by \textbf{three independent groups scattered around the globe} within a very short time.

In the United States, \textbf{H. I. Ewen and E. M. Purcell} were conducting thesis research at Harvard.
They used a large horn antenna ``stuck out a window'' and their initial observations showed neutral hydrogen concentrated towards the galactic plane.
However, their equipment's limitations in resolution and frequency coverage meant they detected only the ``tip of the iceberg'' of the signal, missing its full structure and galactic rotation shifts.

In Holland, \textbf{C.A. Muller and J.H. Oort} were driven to detect the line to study the structure of our galaxy.
They utilized a 7.5 m Wurzburg paraboloid antenna abandoned by the German army.
Their receiver had similar sensitivity to Ewen's, but their frequency-switching interval was also not fully adequate for capturing the full line profiles.
Nevertheless, they successfully measured frequency shifts of the peak intensity at different galactic longitudes.

The third group in Australia, spurred by immediate notification from Ewen and Purcell, quickly assembled equipment and detected the line within months.
In a spirit of scientific cooperation, \textbf{Ewen and Purcell insisted that the initial reports from all three groups ``simultaneously appear''}, which they did in the 1 September 1951 issue of \textit{Nature}.

\section{The Power of the 21 cm Line}

The detection of the 21 cm hydrogen line was a \textbf{transformative event} for astronomy.
Neutral hydrogen is the most abundant element in the universe, and this line provided a unique tool to study its vast clouds between the stars.

Crucially, unlike visible light, radio waves at 21 cm wavelength can \textbf{penetrate the interstellar dust} that obscures much of the Milky Way's structure from optical view.
This capability allowed astronomers to \textbf{map the structure of our own galaxy} in unprecedented detail.
By measuring the Doppler shifts of the 21 cm line, astronomers could determine the velocities of hydrogen clouds, revealing the Milky Way's rotation and spiral structure.
Oort himself had pioneered principles of galactic structure and dynamics that could now be applied using this new tool.
Subsequent large-scale surveys provided the first comprehensive look at this spiral structure.

This period also saw the discovery and initial study of \textbf{discrete radio sources}, often initially called ``radio stars''.
Examples include Cassiopeia A (Cas A), Cygnus A (Cyg A), Taurus A (Tau A), and Virgo A (Vir A).
Pioneers like Hey, Parsons, and Phillips discovered intensity variations in Cyg A, opening up a new line of investigation.
Bolton and Stanley determined that the Cyg A source had a small size.
Later, through painstaking work involving precise position measurements, these sources were optically identified, such as Tau A with the Crab Nebula supernova remnant, Vir A with a galaxy, and most notably, Cyg A with a distant, peculiar galaxy, and Cas A with a new type of galactic emission nebulosity.
The discovery of Cyg A as an extragalactic source had significant implications for cosmology.

Early investigations also included attempts to detect the Sun at radio wavelengths, initially unsuccessfully.
Success came during WWII through the accidental detection by \textbf{James Hey} using radar equipment and independently by \textbf{Grote Southworth}.
These observations showed that the Sun emitted surprisingly strong radio radiation, particularly associated with sunspot activity.
Techniques like using interferometers, analogous to Michelson's optical interferometer, were developed to measure the sizes of sources like the Sun and discrete sources, revealing, for instance, that Cyg A had a complex, possibly double structure.
Other techniques like Dicke switching were developed to improve receiver sensitivity.

The period from the late 19th century attempts to detect solar radio waves through Jansky's discovery and Reber's mapping, culminating in the prediction and detection of the 21 cm hydrogen line, marked a revolutionary era.
It revealed a ``quiescent and limited Universe'' in radio waves, enabling unprecedented studies of our galaxy's structure and paving the way for the exploration of distant cosmic radio sources.
This journey, born from theoretical insight and enabled by improving radio technology and innovative techniques, demonstrates how pioneering work in radio astronomy fundamentally changed our understanding of the universe.

\section{Conclusion}

The journey from theoretical prediction by Van de Hulst under wartime duress, to independent calculations by Shklovsky, and finally, the near-simultaneous detection by three groups across the globe, is a testament to scientific ingenuity and perseverance.
The discovery of the 21 cm hydrogen line, was a watershed moment. It provided astronomers with an indispensable tool for studying the most abundant element in the universe and, critically, for mapping the structure of our own galaxy, a task impossible with optical astronomy alone due to dust obscuration. This breakthrough, born from theoretical insight and enabled by improving radio technology, perfectly embodies the theme of how pioneering work in radio techniques and the interpretation of cosmic signals revolutionized our understanding of the universe. The 21 cm line stands as a prime example of the incredible secrets the radio sky held, waiting to be unlocked.

%\nocite{sullivanClassicsRadioAstronomy1982}
%\nocite{pritchard21CmCosmology2012}
%\nocite{peeblesQuantumMechanics2020}
%\nocite{krausRadioAstronomy1966}
%\nocite{condonEssentialRadioAstronomy2016}
%\bibliography{zotero}
    %</note>
    \printbibliography
\end{document}
