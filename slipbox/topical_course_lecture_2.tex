\documentclass{../template/texnote}

\title{Topical Course Lecture 2}

\begin{document}
    \maketitle \currentdoc{note}
    %<*note>
%Consider gas and radiation at some temperature T.
%The gas and radiation are in thermal equilibrium because the mean free paths are very small.
%This will be derived later.

Let $\ell(\hat{n}, \vb{x}, \nu, t) \mathrm{d}\omega \mathrm{d}\nu$ be the energy per volume at position x and time t\info{A photon can be fully specified by its position and momentum at any instant of time. Momentum can be specified by the frequency and direction of travel.} of photons with directions within a solid angle $d\omega$ around the unit vector $\hat{n}$ and frequencies between $\nu$ and $\nu + d\nu$.
Our first task is to calculate various contributions to the rate of change of $\ell$.
There are four contributions to this rate of change.
These are transport, absorption, scattering, emission (thermal and nuclear).

We can use $\ell$ to define
three fundamental quantities, the radiation energy per volume and per frequency interval
\[
    \epsilon_\nu(\vb{x},t) = \int{\ell_\nu d\Omega}
\]
the flux vector of radiation energy per frequency interval
the stress tensor which is the $i^{th}$ component of radiation force exerted on area element with normal along $j^{th}$ direction per frequency.

    %</note>
    \printbibliography
\end{document}
