\documentclass{../template/texnote}

\title{Mathematical Preliminaries}

\begin{document}
    \maketitle \currentdoc{note}
    %<*note>
    \begin{itemize}
        \item If $f(x)$ is continous in $a \le x \le b $ and if $f(a)$ and $f(b)$ are of opposite signs, then $f(\epsilon) = 0 $ for at least one number $ \epsilon $ such that $ a \le \epsilon b $.
        \item (Maclaurin's expansion) \\
            $ f(x) = f(0) + x f^{\prime}(0) + \frac{x^2}{2!} f^{\prime \prime}(0) + .. + \frac{x^n}{n!}f^{(n)}(0)+ ... $
        \item (Taylors series) \\
            $ f(x) = f(a) + \frac{f^{\prime}(a)(x-a)}{1!} + \frac{f^{\prime\prime}(a)(x-a)^2}{2!}+ ...+ \frac{f^{n}(a)(x-a)^n}{n!}$
        \item (Intermediate value theorem)\\
            Let $f(x)$ be continous in $[a,b]$ and let k be any number between $f(a)$ and $f(b)$. Then there exists a number $\epsilon$ in $(a,b)$ such that $f(\epsilon) = k$.
        \item (Mean value theorem)\\
            If $f(x)$ is continous in $[a,b]$ and $f^{\prime}(x)$ exists in $(a,b)$ then there exists at least one value of x say $\epsilon$ between a and b such that 
            $$ f^{\prime}(x) = \frac{f(b) - f(a)}{b - a}\;\; a \le \epsilon \le b $$
    \end{itemize}
    
    In mathematical analysis,  \\
    The intermediate value theorem states that if a continous function f with an interval $[a,b]$ as its domain takes values $f(a)$ and $f(b)$ at each end of the interval then it also takes any value between $f(a)$ and $f(b)$ at some point within the interval.

    This has two important specializations:

    (Bolzano's Theorem)\\
    If a continous function has values of oppposite sign inside na interval, then it has a root in that interval.
    Both th e intermeidate value theorem and mean value theorem has real world applications.

    The theorem implies that on any great circle around the world, the temperature , pressure, elevation, $C0_2$ cone or any other similar quantity which varies continously there will always exist tow anitpodal points that share the same value for that variable.

    For implications of M.V.T refer Graduate\_stuff.

    \par
    Def:

        If f is integrable on $[a,b]$, its average(mean) value on $[a,b]$ is 
        $$ \frac{1}{b-a} \int_a^b f(x) dx$$
        
        Mean Value Theorem (for definite integrals) \\
        If f is continous on $[a,b]$ then at some point c in $[a,b]$
        $$ f(a) = \frac{1}{b-a}\int_a^b f(x) dx $$

        First Fundamental Theorem of Calculus
        If f is continous on [a,b], then 
        $$ F(x) = \int_a^x f(t) dt $$
        has a derivative at every point and 
        $$ \frac{dF}{dx} = \frac{d}{dx} \int_a^xf(t)dt $$

        Fundamental Theorem of Calculus Part 2,
        $$ \int_a^b f(x) dx = F(b) - F(a) $$
    %</note>
    \printbibliography
\end{document}
