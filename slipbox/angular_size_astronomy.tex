\documentclass{../template/texnote}

\title{Angular Size Astronomy}

\begin{document}
    \maketitle \currentdoc{note}
    %<*note>
\section{Angular sizes in Astronomy}\label{angular-sizes-in-astronomy}

\begin{itemize}
\item
  The apparent size of objects in the sky are often measured using
  angular measure.
\item
  It measures how much of the sky an object covers.
\item
  The apparent size of an object depends both on the actual size of the
  object and the distance to the object.
\item
  All three quantities are related by the small angle approximation.

  \begin{itemize}
  \item
    \[D = \frac{\theta \times d }{206,265''}\]
  \item
    Where D = actual size of the object, d = distance to the object and
    \(\theta\) = angular size of the object in arcsec.
  \item
    Where 206,265 is the number of arcsecs in one radian.
  \end{itemize}
\item
  The explanation is as follows

  \begin{itemize}
  \item
    If theta is the angle subtended at the eye by an astronomical
    object. If theta is in radians, then theta = arc length by radius.
  \item
    Small angle approximation says that for small angles that are
    measured in radians, $\cos \theta = 1$ and $\sin \theta =  \tan \theta = \theta$
  \item
    Or in other words, for very small angles, the arc length can be
    approximated to the diameter or linear size of the object and the
    radius becomes the distance to the object.
  \item
    Hence ,
  \item
    \[\theta = \frac{D}{d} \\
      D = \theta \times d \]
  \item
    If \(\theta\) is measured in arcsecs, you need to divide it by the
    number of arcsecs in 1 radian to convert it to radians.
  \end{itemize}
\item
  The moon covers 1/2 degree in the sky.
\item
  If you extend your hand to arm's length, you can use your fingers to
  estimate angular distances and sizes in the sky. Your index finger is
  about 1° and the distance across your palm is about 10°.
\item
  Refer:
  \url{https://lco.global/spacebook/sky/using-angles-describe-positions-and-apparent-sizes-objects}
\end{itemize}
    %</note>
    \printbibliography
\end{document}
