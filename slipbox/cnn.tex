\documentclass{../template/texnote}

\title{Convolutional Neural Networks}

\begin{document}
    \maketitle \currentdoc{note}
    %<*note>
	\section{Introduction}
	A convolution can be done between a 2D matrix (image) and a 2D matrix (kernel) to produce another 2D matrix (feature map).
	Or it can be done between a 3D matrix (an RGB image, say) and a 3D matrix (a set of three 2D kernels, one for each channel) to produce a 2D matrix as output.
	Note that in both cases, the dimensionality of the output remains two dimensional.
	The dimensionality of the output in general would depend on the dimensions in which we are free to slide the kernel.
	In traditional CNN architecture the kernel is slide in just two dimensions (height and width) but in more advanced networks the kernel can also be slide in the third (channel) dimension.
	This would result in a feature map which has three dimensions.
	Refer here for a more detailed \exhyperref[idea:convolution feature map size]{OutputFeatMapConvolution}{discussion}.

	The working of a CNN can be seen as the combination of multiple affine transformations.\improve{Add reference to note on Affine transformations.}
    %</note>
    \printbibliography
\end{document}
