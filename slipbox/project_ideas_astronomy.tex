\documentclass{../template/texnote}

\title{Project Ideas Astronomy}

\begin{document}
    \maketitle \currentdoc{note}
    %<*note>
	\section{Blackbody Curve}
We see that the spectrum of a lot of astronomical bodies have the shape of the
blackbody spectrum. Verify if this is the case using astronomical data.

\section{Astronomical Distances}
By using Kepler’s law what we get is the distance to the Sun in relative units.
But how do we know the exact value of the Astronomical Unit?
We know that the distances to nearby stars can be found using the method
of parallax. Is it possible to do this using astronomical survey data?

\section{Zeeman Effect}
How to measure Zeeman effect using astronomical data?

\section{Doppler Shift of Spectral Lines}
Where all in astronomy can we see the doppler splitting of spectral lines. Can
we account for this splitting if we know the velocities of these bodies from other
sources? If we know the rotational velocity of the Sun for e.g. can we calculate
the doppler shifting in the lines observed.

\section{Velocities of Galaxies}
We know that as humans we are subject to many different kinds of motion. The
rotation of earth around it’s own axis, that around the Sun. That of the Sun
around the galaxy and the galaxies around something else. Can we measure
any of these velocities from the doppler shifts of lines. For e.g., is it possible
to measure the relative velocity between our galaxy and the Andromeda galaxy
using doppler shift or something else?

    %</note>
    \printbibliography
\end{document}
