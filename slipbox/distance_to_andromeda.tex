\documentclass{../template/texnote}

\title{Distance To Andromeda}

\begin{document}
    \maketitle \currentdoc{note}
    %<*note>

URL: \url{https://www.physlink.com/education/askexperts/ae589.cfm}
\vspace{2 \baselineskip}

\textbf{What is the method used to find out the distance of Andromeda galaxy from us?}
\vspace{2 \baselineskip}

Distances to deep-sky objects such as the Andromeda galaxy are often determined using what are referred to as "Standard Candles", which means that astronomers look for objects located within the target for which they believe they know the intrinsic luminosity (how bright it actually is). For any light-emitting object, it's apparent luminosity (how bright it appears to us) decreases with the square of the distance between the object and the observer; so, if we know the intrinsic luminosity of an object, we can measure the apparent luminosity and perform a straightforward calculation to obtain the approximate distance to the object. An example of such a "Standard Candle" are what are called Cepheid Variable stars. These are young, massive, bright stars (about 1000 times more luminous than our Sun) that undergo periodic changes in luminosity. It has been found that the period of the Cepheid variable is related to its intrinsic brightness, such that if one measures how often the star changes luminosity, one can calculate its intrinsic luminosity, thus allowing the distance to the Cepheid variable to be calculated. Another commonly used "Standard Candle" is a type of supernova, called a Type Ia supernova, caused by the collapse of a white dwarf star that has been stealing mass from a companion star. Astronomers believe that all Type Ia supernovas have approximately the same peak intrinsic luminosity (about -19.5). Again, since the intrinsic luminosity is known, a measurement of the apparent luminosity allows a calculation of the approximate distance to the supernova. Also, since supernovas are so incredibly bright, they can be observed over vast distances, making them ideal for measurements involving objects much further away than the Andromeda galaxy (ie. billions of light-years).

Answered by: Colby Hayward, Computer Support Technician, Ontario, Canada

The distance to Andromeda was first determined (inaccurately) by Edwin Hubble in the late 1920's. Hubble used a calibrated form of the Period Luminosity Relationship first discovered by Henrietta Leavitt around 1911. Leavitt was studying Cepheid variable stars in the much closer galaxy called the Small Magellanic Cloud (SMC). These stars have the peculiar property of varying in brightness in a regular or periodic fashion. The time for a variable star to oscillate in brightness from brightest to dimmest to brightest is called that star's period. Cepheids derive their name from the fact that they were discovered in the constellation Cepheus. Leaviit noticed that their was a linear relationship between the periods of the Cepheids she studied and their apparent brightnesses. Since all these Cepheids were in the SMC, she reasoned that they were all about the same distance away, so there should also be a linear relationship between their periods and their true brightnesses (how bright they would look at a standard distance). It is a simple matter to determine distance to an object if you know its apparent and true brightnesses. You use a law called the Inverse Square Law. Leavitt reasoned that if she could determine the distance to any Cepheid variable star, she could calibrate her linear relationship to determine the distance to any Cepheid variable star. Leavitt's law was calibrated by Harlow Shapley in the late teens, so it was available for use by Hubble about ten years later. The reason that Hubble could use Leavitt's law was that he was at the controls of the new 100 inch Hooker telescope on Mt. Wilson in California. For the first time, he was able to resolve individual stars in the Andromeda Galaxy. Luckily, he found some Cepheids in Andromeda. The important thing about Hubble's work here is that he showed conclusively for the first time that Andromeda was not part of our Milky Way galaxy, but an entirely separate galaxy, and "island universe". Today we know that Andromeda galaxy to be about 2.2 million light years away. Compare this to the radius of the Milky Way, about 50,000 light years, and you will see how phenomenal was his discovery.
Answered by: Robert Mahoney, M.S., President, Magellica Inc.
    %</note>
    \printbibliography
\end{document}
