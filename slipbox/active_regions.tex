\documentclass{../template/texnote}

\title{Active Regions}

\begin{document}
    \maketitle \currentdoc{note}
    %<*note>
	\section{
	\citetitlebyauthor{toriumiFlareproductiveActiveRegions2019}
}

To motivate the definition of an active region, the authors take an example 
NOAA 12192.
This region appeared as one of the largest sunspot groups ever observed with a maximum spot area of 2750 MSH. It also produced numerous solar flares including six X-class events. 

Such centers of activity are called active regions.

Regions with large sunspot groups (count or area?) that can eventually produce flares. 

The active regions can however range from a simple bipole structure to structures consisting of magnetic elements of various size scales as shown in the particular AR.

Flare productivity is known to increase with the “complexity” of the ARs.

    %</note>
    \printbibliography
\end{document}
