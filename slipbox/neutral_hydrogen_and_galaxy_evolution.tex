\documentclass{../template/texnote}

\title{Neutral Hydrogen And Galaxy Evolution}

\begin{document}
    \maketitle \currentdoc{note}
    %<*note>
URL:\url{https://www.youtube.com/live/6pSTGG7aqHs}\\
Speaker: Jacquelin van Gorkom from Columbia University, USA
\begin{itemize}
	\item The 21 cm emission of neutral hydrogen
	\begin{itemize}
		\item This transition is highly ``forbidden''.
		\item It happens spontaneously once every 10 million years.
		\item But fortunately there is a lot of hydrogen in galaxies that we are able to observe the line.
		\item But note that the line is intrinsically weak.
	\end{itemize}
\item The evolution of gas in galaxies.
\item In 1987, they observed a galaxy using radio telescope and saw that in 21cm observations it extended a lot more than it visible extent in optical.
	It was seen to be 13 times the size of the Milky Way.
\item Neutral gas and the evolution of galaxies.
	\begin{itemize}
		\item As galaxies grow over time we want to know: How much HI is inside and around galaxies, where the gas is, and how does it move?
	\end{itemize}
\item 37:16 - In dense clusters, the galaxy types are mostly elliptical.
	And everywhere else the galaxy population is dominated by spirals.
\end{itemize}
    %</note>
    \printbibliography
\end{document}
