\documentclass{../template/texnote}

\title{English Grammar Usage}

\begin{document}
    \maketitle \currentdoc{note}
    %<*note>
	\begin{itemize}
		\item When to use apostrophe?\\
The rule is actually pretty simple: use the apostrophe after it only when part of a word has been removed: it's raining means it is raining; it's been warm means it has been warm. It's is a contraction, in the style of can't for cannot and she's for she is.

\item their's vs theirs \\
their's is incorrect. Its always theirs

\item farther vs further \\
Use farther only when you are referring to distance, literal or figurative. Use further only to mean “more”

\item which or that \\
Use "which" to introduce non-essential or non-restrictive relative clauses. Non-essential clauses provide extra information but can be removed without changing the main meaning of the sentence. Non-essential clauses are set off by commas. Example: "The book, which was published last year, became a bestseller."

\item Use "that" to introduce essential or restrictive relative clauses. Essential clauses provide necessary information that is essential for understanding the main meaning of the sentence. Essential clauses are not set off by commas. Example: "The book that was published last year won an award."

\item PhD \\
In English, PhD can be written with or without periods; both are correct. The trend today is to drop periods with abbreviations of academic degrees. However, many sources, including the Canadian Oxford Dictionary, still recommend the use of periods: Ph.D.

\item to date or till date \\
It is to date, not till date. The correct phrase is ``to date'' because it indicates the date up until which something is true. The phrase ``till date'' would mean the date until which something is true, but the phrase is almost never used that way.

 \item Focused vs focussed \\
Focused, not focussed
Most common spelling is focused according to ngram viewer

\item Dhe or Dha \\
Dhe - before a vowel sound, Dha - before a consonant sound

\item past tense of make do \\
made do

\item catalog vs catalogue \\
Former is American english

\item in order to not create
 or
 in order not to create \\
 ChatGPT says - 
"In order not to create" is the better phrasing. It avoids splitting the infinitive "to create" and follows a more formal writing style.

\item timbre and timber are not same \\
timbre is pronounced as 'tambə'

\item maybe or may be 

	ChatGPT -
\begin{quote}
	The difference between maybe and may be lies in their usage and meaning:

Maybe (one word) is an adverb meaning "perhaps" or "possibly." It’s used to indicate uncertainty or possibility.

Example: "Maybe we'll go to the park tomorrow."

May be (two words) is a verb phrase consisting of the modal verb may and the verb be. It suggests a possibility but is used as part of the sentence’s main verb.

Example: "This may be the answer you're looking for."

A quick test to differentiate: if you can replace it with "perhaps," then maybe (one word) is correct. If "might be" makes more sense, use may be (two words).
\end{quote}
	\end{itemize}
    %</note>
    \printbibliography
\end{document}
