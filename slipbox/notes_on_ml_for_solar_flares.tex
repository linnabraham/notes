\documentclass{../template/texnote}

\title{Notes On ML For Solar Flares}

\begin{document}
    \maketitle \currentdoc{note}
    %<*note>
	\subsection{Discussion}

	\subsection{Problem statement}
	\begin{itemize}
		\item 24 hr or 48h predictability - taken for standardization 
		\item Operational and segmented
		\item A non binary classification approach be feasible with given data. 
	\end{itemize}
	
	\subsection{Data}
	\begin{itemize}
		\item full-disk photospheric vector magnetic field from space
		\item 4 years worth of data.
		\item “that occur within ± 68◦ of the central meridian”
		\item Class imbalance; small no of flares
		\item K-fold cross validation strategies to make maximum use of available data. 
		\item We are limited by observations of actual flares
		\item Does oversampling the minority class work? 
		\item Labelled data a constraint? Data augmentation techniques for neural networks.
		\item Understanding more about nature of data, features and pre-processing
		\item Aditya L1 

	\end{itemize}
	\subsection{Features}
	\begin{itemize}
		\item Multivariate vs Univariate analysis
		\item New features that have come up in literature ever since the paper.
		\item They do pre and post feature selection with f-score analysis and skill score analysis
		\item Better feature selection strategies or dimensionality reduction strategies like PCA 
	\end{itemize}

	\subsection{Network}
	\begin{itemize}
		\item Justification for using a RBF kernel
		\item Neural networks are  seen to be less efficient on extracted feature data as per cited studies.
		\item Possibility of using neural networks on original image data. 

	\end{itemize}
	\subsection{Evaluation metrics / Loss functions.}
	\begin{itemize}
		\item Class imbalance leads to issues of a good metric.
	\item How good of a metric is TSS
	\begin{itemize}
		\item TSS is agnostic to the consequences of the two predictions. Can we weigh the metric to correct for this? 
	\end{itemize}
	\end{itemize}

	\subsection{Other}
	\begin{itemize}
		\item Code availability?
		\item Maintaining the same imbalance ratio for better comparisons.

	\end{itemize}


	\subsection{Quotes}

The small number of flares in our catalog also means that there is an additional risk for our results not to generalize well.

It is not clear what the impact of including or rejecting C-class flares from the catalog is on the performance of the forecasting algorithm. Bloomfield et al. (2012) highlight in their Table 4 that including C-class flares may improve some performance metrics while lowering others: in their case, not including C-class flares increase the true skill statistic (TSS) metric but decrease the Heidke Skill Score
    %</note>
    \printbibliography
\end{document}
