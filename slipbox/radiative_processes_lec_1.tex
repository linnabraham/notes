\documentclass{../template/texnote}

\title{Radiative Processes Lec 1}

\begin{document}
    \maketitle \currentdoc{note}
    %<*note>
\section{Ranjeev Misra Classes on Radiative
Processes}\label{ranjeev-misra-classes-on-radiative-processes}

Ranjeev's notes - \cite{misra_lectures_nodate}

\subsection{Class 1}\label{class-1}

\begin{itemize}
\item
  Hypothesis - Galileo had a hypothesis that the period of the pendulum
  didn't depend on the amplitude.
\item
  Controlled, repeatable - He went on to do a controlled, repeatable
  experiment.
\item
  For a long time, astronomy was not considered as a science.
\item
  The cycle is as follows:

  Observations -\textgreater{} Identify radiative processes
  -\textgreater{} Physical parameters -\textgreater{} Modelling
  (Theorists) -\textgreater{} Predictions -\textgreater{}
  Instrumentation observatory -\textgreater{} Observations
\item
  What is light?

  \begin{itemize}
  \item
    A charged particle being accelerated
  \item
    The power radiated is given by Larmor's formula
  \item
    \[P \propto a^2\]

    \[\propto \left( \frac{F}{m}\right)^2\]
  \end{itemize}
\item
  Since the mass of electron is less than the proton by a factor of
  1000, the power radiated by the electron is greater than the proton by
  a factor of 10\^{}6.
\item
  Hence it is sufficient to talk about the electrons when discussing em
  radiation.
\item
  Electrons can only be accelerated by two things - electric fields or
  magnetic fields
\item
  There are no electric fields in nature on a macroscopic scale
\item
  Electric field due to protons/nuclei

  \begin{itemize}
  \item
    electron bound -\textgreater{} Radiative Transitions
  \item
    electron free

    \begin{itemize}
    \tightlist
    \item
      Brehmsttrahlung - free-free emissions
    \end{itemize}
  \end{itemize}
\item
  Magnetic field can also accelerate electrons

  \begin{itemize}
  \tightlist
  \item
    cyclotron (non-relativistic)
  \item
    Synchrotron ~(relativistic)
  \end{itemize}
\item
  Compton scattering

  \begin{itemize}
  \tightlist
  \item
    Energies change
  \end{itemize}
\item
  Thomson scattering - no change in energy
\end{itemize}

\subsection{Class 2}\label{class-2}

Newton's second law doesn't define force.\\
- But then, how is force defined? Is it given by the universal law of
gravitation which defines what the (gravitational) force is?

The question of which is the more fundamental thing, the force or the
potential has been disregarded as just a philosophical question.

\begin{itemize}
\item
  There are other complications that appear. A moving charge would
  create a changing electric field, which in turn would cause a magnetic
  field. Additionally there is also the Doppler shift which is not the
  relativistic Doppler shift.
\item
  Ranjeev shows the equation for electric fields that even Jackson
  didn't try to derive.
\item
  The direction of the electric field is not from the retarded position
  to the field point but from the present position of the source to the
  field point.
\item
  Note that this is for a charge moving with uniform velocity.
\item
  Acceleration has appeared through the time derivative term associated
  with the vector potential (or magnetic field).
\item
  Ranjeev says that the gamma factor $(1-\beta^{2}$ has appeared without
  any relativistic considerations going in. And he says that this is how
  relativity was developed by Einstein. (On the electrodynamics of
  moving bodies)
\item
  We come to gauge
\end{itemize}

Class 3

\begin{itemize}
\item
  recap

  \begin{itemize}
  \tightlist
  \item
    Electric field had a static and radiative component.
  \item
    E = q n x nxa
  \item
    c\^{}2 is there to make sure the dimensions are correct.
  \item
    Why should electric field have energy?
  \item
    dA/R2 is the solid angle
  \item
    We get the larmor's formula
  \end{itemize}
\item
  if we represent the Electric field as a function of time since a might
  be a function of time,
\item
  frequency is defined as the fourier transform of this.
\item
  What if the acceleration is not a funciton of time? u get 0 freq
  waves.
\item
  Why do we do FT?

  \begin{itemize}
  \tightlist
  \item
    It linearisez the problem
  \item
    it handles derivatives of time veery well.
  \end{itemize}
\item
  Revise, vector prodcuts, FT etc.
\item
  The detector detect the frequency of the radiation.
\item
  Dipole approx

  \begin{itemize}
  \tightlist
  \item
    If v \textless\textless{} c, then phase is not important or phase is
    much less compared to wavelength of radiation?
  \item
    When we ignore phase, we are ignoring when the electrons are
    emitting the radiation.
  \end{itemize}
\item
  net dipole moment of two electrons is zero. no radiation.
\item
  a collection of electrons would not produce radiation. ( a
  relativistic bunch of electrons would emit)
\item
  the accleration is mild. doesnt change the velocity, much.
\item
  QM says that the electron cannot be seen as group of particles like
  jackson assumed it could be treated.
\item
  FT of a constant is zero or delta fun at zero frequency?
\end{itemize}

Class on Sep 03, 2024

\begin{itemize}
\tightlist
\item
  Every emission process has a corresponding absorption process due to
  the time reversal aspect.
\item
  When equilibrium is reached, between emission and absorptoin, When $jv
  = \alpha Bv$
\item
  The =1 is quantum in nature
\item
  -1 comes from stimulated emission.
\item
  In classical there is no concept of two (identical) particles.
\item
  How do magnets attract ? explain using maxwell's eqn and force laws
\item
  When someone announced that physics was over - the known problems was
  the UV divergence issue and the michelson morley experiment.
\item
  So it must mean that Maxwell, understood magnets. How could he explain
  that without QM?
\item
  We have an arbitraty shape and through which we move a distance l
\item
  $Iv(l) = Bv ( 1-exp(-\alpha v l)$
\item
  $jv/\alpha v = B v (blackbody)$
\item
  When we see the blackbody spectrum all the information regarding how
  it was produced is lost. Whether it be through brehmstraulang or
  atomic transitions etc.
\item
  First proof of non-thermal electrons from brightness temperature
  measures in radio astronomy.
\end{itemize}
    %</note>
    \printbibliography
\end{document}
