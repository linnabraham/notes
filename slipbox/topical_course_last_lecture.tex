\documentclass{../template/texnote}

\title{Topical Course Last Lecture}

\begin{document}
    \maketitle \currentdoc{note}
    %<*note>

4th eqn of stellar structure - Temperature gradient required to transport flux through the star
Pressure, nuclear energy epsilon, roseland mean opacity - kappa

Assumptions
Eddington limit - limit on the max luminosity before star starts blowing its own matter.
eta karene - star blowing off
1 - gas pressure dominated
1.6 of weinberg
barren black dimensional analysis
sedovs book on 
pp chain or not lambda (power of rho) is essentially unity
Teff is defined by L eq 4 pi R squared sigma T eff
for higher core temperature u shud use thompson

choice in opacity - thompson or kramers
nuclear energy generation - pp chain or cno
 u might think 4 choices are there but only 3 choices
 for stars like Sun (low mass) - u shud choice kramers and pp-chain 
 Radius is independent of mass ; temperature is roughly prop to M
 for what mass of stars does the CNO cycle dominate? in comparison to Sun's mass
 MESA open source code for stellar evolution
 Luminosity as function of M has a power 5 or something - so that is why when u make approximations
 here getting the numbers correct is important.
 below 1.3 solar mass it is kramers pp ; between 1.3 to 20 solar mass its kramers cno;
 above 20 its thomson cno
 as a star collapses before the temp can get near that required for nuclear fusion
 the electrons become degenerate so that gas pressure becomes independent of temp
 prop to rho power five third (rho - number density) - below 0.8 solar mass
 for high mass the density scales as neg power wrt M - so that high mass stars are puffed up and after wards it get blown off

 eta l divided by eta eff
 L scales as m power 4
 age scales as 1 over m3
    %</note>
    \printbibliography
\end{document}
