\documentclass{../template/texnote}

\title{Black Body Radiation - Understanding
the black body spectra using classical and
quantum physics}

\begin{document}
    \maketitle \currentdoc{note}
    %<*note>

\url{https://www.youtube.com/watch?v=ugsvADj1wts}
\\

2:21 - Temperature is a measure of the average kinetic energy of the electrons.
What this implies is that the kinetic energy of the electrons follow a distribution. That is, some electrons have an energy greater than the average and some lesser.
Or in terms of the velocities of the electrons, some are moving faster than the average and some slower.\info{Think the Maxwell Boltzmann distribution of velocities}
That leads to a distribution of the energy (wavelength) carried by the emitted light.\unsure{But why should an electron that is moving with some velocity emit light unless that electron is accelerating as in vibrations}

The acutal observed shape of the black body radiation cannot be reached from the MB distribution of the electron velocities.
If we try to do that we will land upon the ultra-violet catastrophe.
The relation which we obtain for intensity as a function of the wavelength is known as the Rayleigh-Jeans relation.

The quantum mechanical resolution is to understand that the distribution of the electron velocities as shown by the MB distribution is in reality not continous but quantized.
The kinetic energies that the electrons may have in a material are quantized.
The distribution appears continous because of the very small magnitude of this quantum.
If we add this piece of information we arrive at the Planck's law inplace of the Rayleigh-Jeans relation.

    %</note>
    \printbibliography
\end{document}
