\documentclass{../template/texnote}

\title{Solar Flares}

\begin{document}
    \maketitle \currentdoc{note}
    %<*note>
Flares are impulsive outbursts of electromagnetic energy that takes places occasionally on the Sun's surface. 
These radiate an intense amount of energy across a wide range of frequencies including x-rays. The peak energy of these x-rays as detected by sensors in space can be used to grade these flares into classes like A, B, C, M and X.
Flares emanate from regions of the Sun's surface that are called active regions. These are regions of high concentration of magnetic field lines.
Sunspots, which most people are familiar with, are the photospheric manifestations of these active regions.
It is not necessary that all active regions produce a flare in their lifetime.
In fact, the conditions in an active region that gives rise to flares are not well known. 
Hence it is even more challenging to predict the onset of the flare, provided we know that an active region does flare.
\info{Include properties of ARs that are known to have a relation with flare productivity like the complexity of the magnetic field configuration etc.}
    %</note>
    \printbibliography
\end{document}
