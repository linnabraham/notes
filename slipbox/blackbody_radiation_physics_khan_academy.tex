\documentclass{../template/texnote}

\title{Blackbody Radiation --- Physics --- Khan Academy}

\begin{document}
    \maketitle \currentdoc{note}
    %<*note>

URL: \url{https://www.youtube.com/watch?v=DHG61XPuWyk}
\\

Sunlight is white light\unsure{Is this true? Isn't the peak of the Sunlight near to yellow?} that is composed of all the colors of the rainbow.
When sunlight falls on an apple, the oscillating electric fields in the light would cause the electrons in the apple to oscillate and this would cause the thermal\unsure{What do you mean by thermal here?} energy of the electrons to increase and result in an increase in the temperature of the apple.

But this temperature increase doesn't happen endlessly cause the oscillating electrons produce their own electromagnetic waves (radiation).
These waves are called thermal radiation because that radiation is coming from the thermal\unsure{Thermal again!} motion of the electrons.
At thermal equilibrium, the rate of absorption will equal the rate of emission of radiation and hence the temperature would stay the same.

For most bodies we don't see this thermal radiation as it lies mostly in the infra-red region\unsure{What decides this?}. What we see instead is just the reflected light.
When analyzing thermal radiation, the reflected light is an annoyance.
This is why we dream about bodies which do not reflect any of the light falling on them.
If such bodies did exist and we were asked how they looked, we would say that they looked black, won't we?
So that is why thermal radiation is also called the black body radiation.

The cool thing about this black body radiation is that spectrum only depends on the temperature of the body.
It doesn't depend on what light was shined upon it or what material it was made of.\unsure{Are you sure? Then what differentiates a body that is close to a perfect black body from one that is a poor black body?}

At 5800K even though the peak might be in visible region and around somewhere in the green region say, it wouldn't look green since there is also considerable energy in the other wavelengths. As a result it would look white.
The Sun is actually white not yellow and looks yellow because of the atmosphere.

    %</note>
    \printbibliography
\end{document}
