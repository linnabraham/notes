\documentclass{../template/texnote}

\title{Sunspots}

\begin{document}
    \maketitle \currentdoc{note}
    %<*note>
\section{What are Sunspots?}%\unsure{Question}
Galileo was one of the first people to use a telescope to observe the Sun.
He did this by projecting the Sun's image on a screen.
\footnote{
    For a detailed timeline describing the events relevant to solar astronomy check out Chapter 1 of \citetitlebyauthor{priestMagnetohydrodynamicsSun} and \citetitlebyauthor{choudhuriNaturesThirdCycle2015} for a historical perspective of the events.
}
He understood that these were spots on the surface of the Sun itself and not any intervening bodies.\unsure{How did he understand this?}
These spots then revealed that the Sun's surface didn't rotate uniformly.
But instead, it undergoes what we today call ``differential rotation''.
Which is what one would expect if the Sun was made of gas rather than being a rigid solid body.\unsure{How can this be proven?}

\section{Can the Sun influence events on the Earth?}%\unsure{Question}
People living near the poles often witness bright displays of color on the sky which is known as Aurorae, amongst other names.\unsure{What causes Aurorae and why only near the poles?}
You might already know that the Earth has a magnetic field of it's own\unsure{For whatever reasons that is not clear} and that this is the reason why a freely suspended magnet always points north.
It is now known that this phenomemon of Aurorae is a result of the interaction of Earth's magnetic field with streams of charged particles\unsure{Is it?} that come from the Sun.
An extreme version of this happened in the September of 1859 when, Richard Carrington, an English astronomer observed what would be the most powerful ``solar flare'' ever to be observed.
\unsure{What is a flare?}
%\improve{What did Carrington actually see? Add this here so that it can serve as an explanation for what a flare is.}
For more information on what Carrington actually witnessed and  what he didn't, refer to Chapter 1 of \citetitlebyauthor{choudhuriNaturesThirdCycle2015}.
Flares on the Sun happen frequently. It is the scale of the event that matters and decides their influence on the Earth.

\section{Sunspots and Nature's Third Cycle}
The sunspots come in different sizes \unsure{What is the size of a typical sunspot?} and shapes.
Also they have a fixed lifetime after which they would disappear and new spots would form.
% it is said to be a common theme that people understand the least about things that are the closest.
Astronomers decided to keep count of these sunspots and amassed this data over a long period.
Analyzing this data it was found that there is a cyclic pattern in the appearance and disappearance of these sunspots.
And that this cycle repeated with a period of 11 years.
When plotted along with the latitude at which these spots appear, they result in the butterfly diagram that most solar astronomers are familiar with.\unsure{How can we explain this solar cycle and also the period?}

\section{Hale's discovery}
George Hale in 1980 discovered that sunspots were areas of intense magnetic field.
The principle that helps one to come to such a conclusion is well known to any student of physics and is called the Zeeman effect.\unsure{How is Zeeman effect used to understand this?}
\unsure{Did people understand how the Sun would possess a magnetic field back then?}
\improve{Add section on Magnetogram maps of the Sun}

    %</note>
    \printbibliography
\end{document}
