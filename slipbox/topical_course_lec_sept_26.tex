\documentclass{../template/texnote}

\title{Topical Course Lec Sept 26}

\begin{document}
    \maketitle \currentdoc{note}
    %<*note>
Free - Free emission
energy emitted per vol per freq, which is rho j nu
equal to ni times int from vmin to inf dnu integral bmin to inf db Enu R(b,v)

In classical picture, for a partiuclar freq, you would that you can have as low an energy of radiation as possible but qm at least one photon needs to be there.
the spin angular momentum of electrons is either plus h cross or minus h cross.
which is same as saying that there are only two polarizations.

in the limit of weak scattering
1/2 mvmin square = hnu
mvb > hcross
theta approx delta v by b approx at by v less than 10
bmin is set by these two conditions.
b > Zesquare by mvsquare
is the larger of these two.


At LTE,
rho Kappa nu superscript ff(free-free) = 

Integrate over frequencies to get Rosseland mean opactities.(for free free emission)
Krammers opacity
Krammer did the first classical treatment of free free opacity.
Thomson opacity - what are values of alpha and beta.

Sridhar asked, 
if lets say  in the lecture room where it is much more cooler than the Sun is this eqn true? 
Is there any contribution from free free emission?

There is a temperature dependence on ne which comes from Saha's ionization formula.
this is true if hot enough T > 3000K
else there wont be much ionization.

    %</note>
    \printbibliography
\end{document}
