\documentclass{../template/texnote}

\title{Hypothesis Testing And The Null Hypothesis, Clearly Explained!!!}

\begin{document}
    \maketitle \currentdoc{note}
    %<*note>
	URL - \url{https://youtu.be/0oc49DyA3hU} - by StatQuest
    \begin{itemize}
        \item The (null) hypothesis is made on the basis of some preliminary investigation.
        \item If data gives us strong evidence that the hypothesis is wrong, then we can reject the hypothesis.
        \item But when we have data that is similar to the hypothesis but not exactly same, then the best we can do is fail 
            to reject the hypothesis.
        \item Because, then it's unclear if the hypothesis should be based on this initial result (which prompted you to make the hypothesis) or 
            these other slightly different results.
        \item The hypothesis that there is no difference between things is the Null Hypothesis.
        \item We always start with the null hypothesis since without it we would need to do some background research to test a specific number.
        \item There is a follow up to this video - Title: Alternative Hypotheses: Main Ideas!!!, where one idea I got is that, failing to reject the hull
            hypothesis in favour of an alternate hypothesis doesn't mean the same as accepting that alternate hypothesis.

    \end{itemize}
    %</note>
    \printbibliography
\end{document}
