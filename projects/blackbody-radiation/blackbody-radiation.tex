
\documentclass{book} 

\usepackage{amsmath}
\usepackage{amsthm}
\usepackage{amssymb}
\usepackage{mathrsfs}
\usepackage{slashed}


\usepackage{hyperref} 
\usepackage[capitalize,nameinlink,noabbrev]{cleveref}


\usepackage{slashed}
\usepackage{xargs}
\usepackage[dvipsnames]{xcolor}
\usepackage[colorinlistoftodos,prependcaption,textsize=tiny]{todonotes}




\usepackage[style=alphabetic]{biblatex}
\addbibresource{../../../bibliography.bib}


\hypersetup{
    colorlinks = true,
    citecolor = cyan
}

%Theorem environment definitions
\theoremstyle{definition}
\newtheorem{definition}{Definition}[section]

\newtheorem{example}[definition]{Example}
\newtheorem{prop}[definition]{Proposition}
\newtheorem{lemma}[definition]{Lemma}
\newtheorem{thm}[definition]{Theorem}
\newtheorem{cor}[definition]{Corollary}
\newtheorem{rmk}[definition]{Remark}
\newtheorem{conj}[definition]{Conjecture}

%Change the excref and exhyperref commands back to normal
\newcommand{\excref}[2][note]{\cref{#1}}
\newcommand{\exhyperref}[3][note]{\hyperref[#1]{#3}}
\newcommand{\transclude}[2][note]{\ExecuteMetaData[../../slipbox/#2.tex]{#1}}

\newcommandx{\info}[2][1=]{\todo[linecolor=violet,backgroundcolor=violet!25,bordercolor=violet,#1]{#2}}
\newcommandx{\change}[2][1=]{\todo[linecolor=blue,backgroundcolor=blue!25,bordercolor=blue,#1]{#2}}
\newcommandx{\unsure}[2][1=]{\todo[linecolor=OliveGreen,backgroundcolor=OliveGreen!25,bordercolor=OliveGreen,#1]{#2}}
\newcommandx{\improve}[2][1=]{\todo[linecolor=Plum,backgroundcolor=Plum!25,bordercolor=Plum,#1]{#2}}
\newcommandx{\thiswillnotshow}[2][1=]{\todo[disable,#1]{#2}}

% have title and author's full name appear in in-text citation
\newrobustcmd*{\citefirstlastauthor}{\AtNextCite{\DeclareNameAlias{labelname}{given-family}}\citeauthor}
\newcommand{\citetitlebyauthor}[2][]{
    \citetitle{#2} by \citefirstlastauthor[#1]{#2}
}
\usepackage{catchfilebetweentags}

%Fix newlines with catchfilebetweentags
\usepackage{etoolbox}
\makeatletter
\patchcmd{\CatchFBT@Fin@l}{\endlinechar\m@ne}{}
{}{\typeout{Unsuccessful patch!}}
\makeatother

\title{Notes on Black body radiation}



\begin{document}
    \maketitle

	\chapter{Why Do Hot Objects Glow - Black Body Radiation}

    %Import all of a note
    \transclude{why_do_hot_objects_glow}
    %The note argument of ExecuteMetaData corresponds to the <*note>, </note> tags in the note template. 
    %To import only part of a note then you can surround that part by a different tag, say <*part>...</part> and then call

	\chapter{Blackbody Radiation --- Physics --- Khan Academy}
    \transclude{blackbody_radiation_physics_khan_academy}
	\chapter{What Is A Black body (Stefan Boltzmann’s Law, emissivity, grey and white bodies...) - Physics}
    \transclude{blackbody_radiation_physics_khan_academy}
	\chapter{Black Body Radiation - Understanding
the black body spectra using classical and
quantum physics}
    \transclude{black_body_radiation_classical_quantum}
	\chapter{Quantization Of Energy Part 1: Black-body Radiation and the Ultraviolet Catastrophe}

    \transclude{quantization_of_energy_part_1}
	\chapter{Bowley Sanchez Chapter 7}
    \transclude{bowley_sanchez_chapter_7}
	\chapter{Bowley Sanchez Chapter 8}
    \transclude{bowley_sanchez_chapter_8}
	\chapter{My Thoughts}
	\transclude{black_body_radiation}
     

    \printbibliography

\end{document}
